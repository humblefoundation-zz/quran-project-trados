%This file is used by a ruby script as a template for each aayah
\begin{comment}
The following strings are to be replaced by a script, in order to use this file as a template (all upper case):-
* sN = sūrah number, without leading zeros
* aYN = āyah number, without leading zeros
* aRABIC_TEXT_TASHKEEL = the text of the aayah, with tashkeel marks
* aRABIC_TEXT_WITHOUT_TASHKEEL = the text of the aayah, without tashkeel marks
* tAFSEER_SADI_ARABIC = the tafseer of the aayah from as-sa'di
\end{comment}
\begin{comment}
The following tags are declared here:-
quran_com_link
quran_com_arabic_image
arabic_text_tashkeel
our_translation
variations_in_reading
translator_comments
aayah_tags
tafseer_sadi_arabic
tafseer_sadi_translation
\end{comment}
\begin{taggedblock}{quran_com_link}
\href{http://quran.com/1/2}{Link to 1:2 on quran.com}
\end{taggedblock}
\begin{taggedblock}{quran_com_arabic_image}
\includegraphics{1_2}
\end{taggedblock}
\begin{taggedblock}{arabic_text_tashkeel}
\begin{Arabic}
الْحَمْدُ لِلَّهِ رَبِّ الْعَالَمِينَ
\end{Arabic}
\end{taggedblock}
\begin{taggedblock}{our_translation}
الحمد لله رب العالمين
\end{taggedblock}
\begin{taggedblock}{variations_in_reading}
%This section is optional, for translating different wordings. For each different wording, paste the translation again, with the changes from Hafṣ highlighted in bold.
\end{taggedblock}
\begin{taggedblock}{translator_comments}
%Put any comments that you have as a translator, including issues of concern, or major decisions that you made when translating.
\end{taggedblock}
\begin{taggedblock}{aayah_tags}
%Put tags here separated by commas, e.g.: tawheed,prophets,yusuf,dua
\end{taggedblock}
\begin{taggedblock}{tafseer_sadi_arabic}
\begin{Arabic}
{ الْحَمْدُ لِلَّهِ }

[هو]
الثناء على الله بصفات الكمال, وبأفعاله الدائرة بين الفضل والعدل, فله الحمد الكامل, بجميع الوجوه.
{ رَبِّ الْعَالَمِينَ }
الرب, هو المربي جميع العالمين -وهم من سوى الله- بخلقه إياهم, وإعداده لهم الآلات, وإنعامه عليهم بالنعم العظيمة, التي لو فقدوها, لم يمكن لهم البقاء. فما بهم من نعمة, فمنه تعالى. وتربيته تعالى لخلقه نوعان: عامة وخاصة. فالعامة: هي خلقه للمخلوقين, ورزقهم, وهدايتهم لما فيه مصالحهم, التي فيها بقاؤهم في الدنيا. والخاصة: تربيته لأوليائه, فيربيهم بالإيمان, ويوفقهم له, ويكمله لهم, ويدفع عنهم الصوارف, والعوائق الحائلة بينهم وبينه, وحقيقتها: تربية التوفيق لكل خير, والعصمة عن كل شر. ولعل هذا
[المعنى]
هو السر في كون أكثر أدعية الأنبياء بلفظ الرب. فإن مطالبهم كلها داخلة تحت ربوبيته الخاصة. فدل قوله
{ رَبِّ الْعَالَمِينَ }
على انفراده بالخلق والتدبير, والنعم, وكمال غناه, وتمام فقر العالمين إليه, بكل وجه واعتبار.
\end{Arabic}
\end{taggedblock}
\begin{taggedblock}{tafseer_sadi_translation}
{ الْحَمْدُ لِلَّهِ }

[هو]
الثناء على الله بصفات الكمال, وبأفعاله الدائرة بين الفضل والعدل, فله الحمد الكامل, بجميع الوجوه.
{ رَبِّ الْعَالَمِينَ }
الرب, هو المربي جميع العالمين -وهم من سوى الله- بخلقه إياهم, وإعداده لهم الآلات, وإنعامه عليهم بالنعم العظيمة, التي لو فقدوها, لم يمكن لهم البقاء. فما بهم من نعمة, فمنه تعالى. وتربيته تعالى لخلقه نوعان: عامة وخاصة. فالعامة: هي خلقه للمخلوقين, ورزقهم, وهدايتهم لما فيه مصالحهم, التي فيها بقاؤهم في الدنيا. والخاصة: تربيته لأوليائه, فيربيهم بالإيمان, ويوفقهم له, ويكمله لهم, ويدفع عنهم الصوارف, والعوائق الحائلة بينهم وبينه, وحقيقتها: تربية التوفيق لكل خير, والعصمة عن كل شر. ولعل هذا
[المعنى]
هو السر في كون أكثر أدعية الأنبياء بلفظ الرب. فإن مطالبهم كلها داخلة تحت ربوبيته الخاصة. فدل قوله
{ رَبِّ الْعَالَمِينَ }
على انفراده بالخلق والتدبير, والنعم, وكمال غناه, وتمام فقر العالمين إليه, بكل وجه واعتبار.
\end{taggedblock}
\input{license-attribution}
\begin{comment}
Please use the following for footnotes:- Sample\footnoteQ{Text of Qur'an footnote goes here.}.
Sample\footnoteT{Text of Tafseer footnote goes here.}.
\end{comment}