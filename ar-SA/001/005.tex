%This file is used by a ruby script as a template for each aayah
\begin{comment}
The following strings are to be replaced by a script, in order to use this file as a template (all upper case):-
* sN = sūrah number, without leading zeros
* aYN = āyah number, without leading zeros
* aRABIC_TEXT_TASHKEEL = the text of the aayah, with tashkeel marks
* aRABIC_TEXT_WITHOUT_TASHKEEL = the text of the aayah, without tashkeel marks
* tAFSEER_SADI_ARABIC = the tafseer of the aayah from as-sa'di
\end{comment}
\begin{comment}
The following tags are declared here:-
quran_com_link
quran_com_arabic_image
arabic_text_tashkeel
our_translation
variations_in_reading
translator_comments
aayah_tags
tafseer_sadi_arabic
tafseer_sadi_translation
\end{comment}
\begin{taggedblock}{quran_com_link}
\href{http://quran.com/1/5}{Link to 1:5 on quran.com}
\end{taggedblock}
\begin{taggedblock}{quran_com_arabic_image}
\includegraphics{1_5}
\end{taggedblock}
\begin{taggedblock}{arabic_text_tashkeel}
\begin{Arabic}
إِيَّاكَ نَعْبُدُ وَإِيَّاكَ نَسْتَعِينُ
\end{Arabic}
\end{taggedblock}
\begin{taggedblock}{our_translation}
إياك نعبد وإياك نستعين
\end{taggedblock}
\begin{taggedblock}{variations_in_reading}
%This section is optional, for translating different wordings. For each different wording, paste the translation again, with the changes from Hafṣ highlighted in bold.
\end{taggedblock}
\begin{taggedblock}{translator_comments}
%Put any comments that you have as a translator, including issues of concern, or major decisions that you made when translating.
\end{taggedblock}
\begin{taggedblock}{aayah_tags}
%Put tags here separated by commas, e.g.: tawheed,prophets,yusuf,dua
\end{taggedblock}
\begin{taggedblock}{tafseer_sadi_arabic}
\begin{Arabic}
وقوله
{ إِيَّاكَ نَعْبُدُ وَإِيَّاكَ نَسْتَعِينُ }
أي: نخصك وحدك بالعبادة والاستعانة, لأن تقديم المعمول يفيد الحصر, وهو إثبات الحكم للمذكور, ونفيه عما عداه. فكأنه يقول: نعبدك, ولا نعبد غيرك, ونستعين بك, ولا نستعين بغيرك. وقدم العبادة على الاستعانة, من باب تقديم العام على الخاص, واهتماما بتقديم حقه تعالى على حق عبده. و
{ العبادة }
اسم جامع لكل ما يحبه الله ويرضاه من الأعمال, والأقوال الظاهرة والباطنة. و
{ الاستعانة }
هي الاعتماد على الله تعالى في جلب المنافع, ودفع المضار, مع الثقة به في تحصيل ذلك. والقيام بعبادة الله والاستعانة به هو الوسيلة للسعادة الأبدية, والنجاة من جميع الشرور, فلا سبيل إلى النجاة إلا بالقيام بهما. وإنما تكون العبادة عبادة, إذا كانت مأخوذة عن رسول الله صلى الله عليه وسلم مقصودا بها وجه الله. فبهذين الأمرين تكون عبادة, وذكر
{ الاستعانة }
بعد
{ العبادة }
مع دخولها فيها, لاحتياج العبد في جميع عباداته إلى الاستعانة بالله تعالى. فإنه إن لم يعنه الله, لم يحصل له ما يريده من فعل الأوامر, واجتناب النواهي.
\end{Arabic}
\end{taggedblock}
\begin{taggedblock}{tafseer_sadi_translation}
وقوله
{ إِيَّاكَ نَعْبُدُ وَإِيَّاكَ نَسْتَعِينُ }
أي: نخصك وحدك بالعبادة والاستعانة, لأن تقديم المعمول يفيد الحصر, وهو إثبات الحكم للمذكور, ونفيه عما عداه. فكأنه يقول: نعبدك, ولا نعبد غيرك, ونستعين بك, ولا نستعين بغيرك. وقدم العبادة على الاستعانة, من باب تقديم العام على الخاص, واهتماما بتقديم حقه تعالى على حق عبده. و
{ العبادة }
اسم جامع لكل ما يحبه الله ويرضاه من الأعمال, والأقوال الظاهرة والباطنة. و
{ الاستعانة }
هي الاعتماد على الله تعالى في جلب المنافع, ودفع المضار, مع الثقة به في تحصيل ذلك. والقيام بعبادة الله والاستعانة به هو الوسيلة للسعادة الأبدية, والنجاة من جميع الشرور, فلا سبيل إلى النجاة إلا بالقيام بهما. وإنما تكون العبادة عبادة, إذا كانت مأخوذة عن رسول الله صلى الله عليه وسلم مقصودا بها وجه الله. فبهذين الأمرين تكون عبادة, وذكر
{ الاستعانة }
بعد
{ العبادة }
مع دخولها فيها, لاحتياج العبد في جميع عباداته إلى الاستعانة بالله تعالى. فإنه إن لم يعنه الله, لم يحصل له ما يريده من فعل الأوامر, واجتناب النواهي.
\end{taggedblock}
\input{license-attribution}
\begin{comment}
Please use the following for footnotes:- Sample\footnoteQ{Text of Qur'an footnote goes here.}.
Sample\footnoteT{Text of Tafseer footnote goes here.}.
\end{comment}