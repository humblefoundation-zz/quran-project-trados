%This file is used by a ruby script as a template for each aayah
\begin{comment}
The following strings are to be replaced by a script, in order to use this file as a template (all upper case):-
* sN = sūrah number, without leading zeros
* aYN = āyah number, without leading zeros
* aRABIC_TEXT_TASHKEEL = the text of the aayah, with tashkeel marks
* aRABIC_TEXT_WITHOUT_TASHKEEL = the text of the aayah, without tashkeel marks
* tAFSEER_SADI_ARABIC = the tafseer of the aayah from as-sa'di
\end{comment}
\begin{comment}
The following tags are declared here:-
quran_com_link
quran_com_arabic_image
arabic_text_tashkeel
our_translation
variations_in_reading
translator_comments
aayah_tags
tafseer_sadi_arabic
tafseer_sadi_translation
\end{comment}
\begin{taggedblock}{quran_com_link}
\href{http://quran.com/1/7}{Link to 1:7 on quran.com}
\end{taggedblock}
\begin{taggedblock}{quran_com_arabic_image}
\includegraphics{1_7}
\end{taggedblock}
\begin{taggedblock}{arabic_text_tashkeel}
\begin{Arabic}
صِرَاطَ الَّذِينَ أَنْعَمْتَ عَلَيْهِمْ غَيْرِ الْمَغْضُوبِ عَلَيْهِمْ وَلَا الضَّالِّينَ
\end{Arabic}
\end{taggedblock}
\begin{taggedblock}{our_translation}
صراط الذين أنعمت عليهم غير المغضوب عليهم ولا الضالين
\end{taggedblock}
\begin{taggedblock}{variations_in_reading}
%This section is optional, for translating different wordings. For each different wording, paste the translation again, with the changes from Hafṣ highlighted in bold.
\end{taggedblock}
\begin{taggedblock}{translator_comments}
%Put any comments that you have as a translator, including issues of concern, or major decisions that you made when translating.
\end{taggedblock}
\begin{taggedblock}{aayah_tags}
%Put tags here separated by commas, e.g.: tawheed,prophets,yusuf,dua
\end{taggedblock}
\begin{taggedblock}{tafseer_sadi_arabic}
\begin{Arabic}
وهذا الصراط المستقيم هو:
{ صِرَاطَ الَّذِينَ أَنْعَمْتَ عَلَيْهِمْ }
من النبيين والصديقين والشهداء والصالحين.
{ غَيْرِ }
صراط
{ الْمَغْضُوبِ عَلَيْهِمْ }
الذين عرفوا الحق وتركوه كاليهود ونحوهم. وغير صراط
{ الضَّالِّينَ }
الذين تركوا الحق على جهل وضلال, كالنصارى ونحوهم. فهذه السورة على إيجازها, قد احتوت على ما لم تحتو عليه سورة من سور القرآن, فتضمنت أنواع التوحيد الثلاثة: توحيد الربوبية يؤخذ من قوله:
{ رَبِّ الْعَالَمِينَ }
وتوحيد الإلهية وهو إفراد الله بالعبادة, يؤخذ من لفظ:
{ اللَّهِ }
ومن قوله:
{ إِيَّاكَ نَعْبُدُ }
وتوحيد الأسماء والصفات, وهو إثبات صفات الكمال لله تعالى, التي أثبتها لنفسه, وأثبتها له رسوله من غير تعطيل ولا تمثيل ولا تشبيه, وقد دل على ذلك لفظ
{ الْحَمْدُ }
كما تقدم. وتضمنت إثبات النبوة في قوله:
{ اهْدِنَا الصِّرَاطَ الْمُسْتَقِيمَ }
لأن ذلك ممتنع بدون الرسالة. وإثبات الجزاء على الأعمال في قوله:
{ مَالِكِ يَوْمِ الدِّينِ }
وأن الجزاء يكون بالعدل, لأن الدين معناه الجزاء بالعدل. وتضمنت إثبات القدر, وأن العبد فاعل حقيقة, خلافا للقدرية والجبرية. بل تضمنت الرد على جميع أهل البدع
[والضلال]
في قوله:
{ اهْدِنَا الصِّرَاطَ الْمُسْتَقِيمَ }
لأنه معرفة الحق والعمل به. وكل مبتدع
[وضال]
فهو مخالف لذلك. وتضمنت إخلاص الدين لله تعالى, عبادة واستعانة في قوله:
{ إِيَّاكَ نَعْبُدُ وَإِيَّاكَ نَسْتَعِينُ }
فالحمد لله رب العالمين.
\end{Arabic}
\end{taggedblock}
\begin{taggedblock}{tafseer_sadi_translation}
وهذا الصراط المستقيم هو:
{ صِرَاطَ الَّذِينَ أَنْعَمْتَ عَلَيْهِمْ }
من النبيين والصديقين والشهداء والصالحين.
{ غَيْرِ }
صراط
{ الْمَغْضُوبِ عَلَيْهِمْ }
الذين عرفوا الحق وتركوه كاليهود ونحوهم. وغير صراط
{ الضَّالِّينَ }
الذين تركوا الحق على جهل وضلال, كالنصارى ونحوهم. فهذه السورة على إيجازها, قد احتوت على ما لم تحتو عليه سورة من سور القرآن, فتضمنت أنواع التوحيد الثلاثة: توحيد الربوبية يؤخذ من قوله:
{ رَبِّ الْعَالَمِينَ }
وتوحيد الإلهية وهو إفراد الله بالعبادة, يؤخذ من لفظ:
{ اللَّهِ }
ومن قوله:
{ إِيَّاكَ نَعْبُدُ }
وتوحيد الأسماء والصفات, وهو إثبات صفات الكمال لله تعالى, التي أثبتها لنفسه, وأثبتها له رسوله من غير تعطيل ولا تمثيل ولا تشبيه, وقد دل على ذلك لفظ
{ الْحَمْدُ }
كما تقدم. وتضمنت إثبات النبوة في قوله:
{ اهْدِنَا الصِّرَاطَ الْمُسْتَقِيمَ }
لأن ذلك ممتنع بدون الرسالة. وإثبات الجزاء على الأعمال في قوله:
{ مَالِكِ يَوْمِ الدِّينِ }
وأن الجزاء يكون بالعدل, لأن الدين معناه الجزاء بالعدل. وتضمنت إثبات القدر, وأن العبد فاعل حقيقة, خلافا للقدرية والجبرية. بل تضمنت الرد على جميع أهل البدع
[والضلال]
في قوله:
{ اهْدِنَا الصِّرَاطَ الْمُسْتَقِيمَ }
لأنه معرفة الحق والعمل به. وكل مبتدع
[وضال]
فهو مخالف لذلك. وتضمنت إخلاص الدين لله تعالى, عبادة واستعانة في قوله:
{ إِيَّاكَ نَعْبُدُ وَإِيَّاكَ نَسْتَعِينُ }
فالحمد لله رب العالمين.
\end{taggedblock}
\input{license-attribution}
\begin{comment}
Please use the following for footnotes:- Sample\footnoteQ{Text of Qur'an footnote goes here.}.
Sample\footnoteT{Text of Tafseer footnote goes here.}.
\end{comment}