%This file is used by a ruby script as a template for each aayah
\begin{comment}
The following strings are to be replaced by a script, in order to use this file as a template (all upper case):-
* sN = sūrah number, without leading zeros
* aYN = āyah number, without leading zeros
* aRABIC_TEXT_TASHKEEL = the text of the aayah, with tashkeel marks
* aRABIC_TEXT_WITHOUT_TASHKEEL = the text of the aayah, without tashkeel marks
* tAFSEER_SADI_ARABIC = the tafseer of the aayah from as-sa'di
\end{comment}
\begin{comment}
The following tags are declared here:-
quran_com_link
quran_com_arabic_image
arabic_text_tashkeel
our_translation
variations_in_reading
translator_comments
aayah_tags
tafseer_sadi_arabic
tafseer_sadi_translation
license_attribution_aayah
\end{comment}
\begin{taggedblock}{quran_com_link}
\href{http://quran.com/1/1}{Link to 1:1 on quran.com}
\end{taggedblock}
\begin{taggedblock}{quran_com_arabic_image}
\includegraphics{1_1}
\end{taggedblock}
\begin{taggedblock}{arabic_text_tashkeel}
\begin{Arabic}
بِسْمِ اللَّهِ الرَّحْمَٰنِ الرَّحِيمِ
\end{Arabic}
\end{taggedblock}
\begin{taggedblock}{our_translation}
بسم الله الرحمن الرحيم
\end{taggedblock}
\begin{taggedblock}{variations_in_reading}
%This section is optional, for translating different wordings. For each different wording, paste the translation again, with the changes from Hafṣ highlighted in bold.
\end{taggedblock}
\begin{taggedblock}{translator_comments}
%Put any comments that you have as a translator, including issues of concern, or major decisions that you made when translating.
\end{taggedblock}
\begin{taggedblock}{aayah_tags}
%Put tags here separated by commas, e.g.: tawheed,prophets,yusuf,dua
\end{taggedblock}
\begin{taggedblock}{tafseer_sadi_arabic}
\begin{Arabic}
{ بِسْمِ اللَّهِ }
أي: أبتدئ بكل اسم لله تعالى, لأن لفظ
{ اسم }
مفرد مضاف, فيعم جميع الأسماء
[الحسنى]
.
{ اللَّهِ }
هو المألوه المعبود, المستحق لإفراده بالعبادة, لما اتصف به من صفات الألوهية وهي صفات الكمال.
{ الرَّحْمَنِ الرَّحِيمِ }
اسمان دالان على أنه تعالى ذو الرحمة الواسعة العظيمة التي وسعت كل شيء, وعمت كل حي, وكتبها للمتقين المتبعين لأنبيائه ورسله. فهؤلاء لهم الرحمة المطلقة, ومن عداهم فلهم نصيب منها. واعلم أن من القواعد المتفق عليها بين سلف الأمة وأئمتها, الإيمان بأسماء الله وصفاته, وأحكام الصفات. فيؤمنون مثلا, بأنه رحمن رحيم, ذو الرحمة التي اتصف بها, المتعلقة بالمرحوم. فالنعم كلها, أثر من آثار رحمته, وهكذا في سائر الأسماء. يقال في العليم: إنه عليم ذو علم, يعلم
[به]
كل شيء, قدير, ذو قدرة يقدر على كل شيء.
\end{Arabic}
\end{taggedblock}
\begin{taggedblock}{tafseer_sadi_translation}
{ بِسْمِ اللَّهِ }
أي: أبتدئ بكل اسم لله تعالى, لأن لفظ
{ اسم }
مفرد مضاف, فيعم جميع الأسماء
[الحسنى]
.
{ اللَّهِ }
هو المألوه المعبود, المستحق لإفراده بالعبادة, لما اتصف به من صفات الألوهية وهي صفات الكمال.
{ الرَّحْمَنِ الرَّحِيمِ }
اسمان دالان على أنه تعالى ذو الرحمة الواسعة العظيمة التي وسعت كل شيء, وعمت كل حي, وكتبها للمتقين المتبعين لأنبيائه ورسله. فهؤلاء لهم الرحمة المطلقة, ومن عداهم فلهم نصيب منها. واعلم أن من القواعد المتفق عليها بين سلف الأمة وأئمتها, الإيمان بأسماء الله وصفاته, وأحكام الصفات. فيؤمنون مثلا, بأنه رحمن رحيم, ذو الرحمة التي اتصف بها, المتعلقة بالمرحوم. فالنعم كلها, أثر من آثار رحمته, وهكذا في سائر الأسماء. يقال في العليم: إنه عليم ذو علم, يعلم
[به]
كل شيء, قدير, ذو قدرة يقدر على كل شيء.
\end{taggedblock}
\begin{taggedblock}{license_attribution_aayah}
\input{license-attribution}
\end{taggedblock}
\begin{comment}
Please use the following for footnotes:- Sample\footnoteQ{Text of Qur'an footnote goes here.}.
Sample\footnoteT{Text of Tafseer footnote goes here.}.
\end{comment}