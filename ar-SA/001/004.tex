%This file is used by a ruby script as a template for each aayah
\begin{comment}
The following strings are to be replaced by a script, in order to use this file as a template (all upper case):-
* sN = sūrah number, without leading zeros
* aYN = āyah number, without leading zeros
* aRABIC_TEXT_TASHKEEL = the text of the aayah, with tashkeel marks
* aRABIC_TEXT_WITHOUT_TASHKEEL = the text of the aayah, without tashkeel marks
* tAFSEER_SADI_ARABIC = the tafseer of the aayah from as-sa'di
\end{comment}
\begin{comment}
The following tags are declared here:-
quran_com_link
quran_com_arabic_image
arabic_text_tashkeel
our_translation
variations_in_reading
translator_comments
aayah_tags
tafseer_sadi_arabic
tafseer_sadi_translation
license_attribution_aayah
\end{comment}
\begin{taggedblock}{quran_com_link}
\href{http://quran.com/1/4}{Link to 1:4 on quran.com}
\end{taggedblock}
\begin{taggedblock}{quran_com_arabic_image}
\includegraphics{1_4}
\end{taggedblock}
\begin{taggedblock}{arabic_text_tashkeel}
\begin{Arabic}
مَالِكِ يَوْمِ الدِّينِ
\end{Arabic}
\end{taggedblock}
\begin{taggedblock}{our_translation}
مالك يوم الدين
\end{taggedblock}
\begin{taggedblock}{variations_in_reading}
%This section is optional, for translating different wordings. For each different wording, paste the translation again, with the changes from Hafṣ highlighted in bold.
\end{taggedblock}
\begin{taggedblock}{translator_comments}
%Put any comments that you have as a translator, including issues of concern, or major decisions that you made when translating.
\end{taggedblock}
\begin{taggedblock}{aayah_tags}
%Put tags here separated by commas, e.g.: tawheed,prophets,yusuf,dua
\end{taggedblock}
\begin{taggedblock}{tafseer_sadi_arabic}
\begin{Arabic}
{ مَالِكِ يَوْمِ الدِّينِ }
المالك: هو من اتصف بصفة الملك التي من آثارها أنه يأمر وينهى, ويثيب ويعاقب, ويتصرف بمماليكه بجميع أنواع التصرفات, وأضاف الملك ليوم الدين, وهو يوم القيامة, يوم يدان الناس فيه بأعمالهم, خيرها وشرها, لأن في ذلك اليوم, يظهر للخلق تمام الظهور, كمال ملكه وعدله وحكمته, وانقطاع أملاك الخلائق. حتى
[إنه]
يستوي في ذلك اليوم, الملوك والرعايا والعبيد والأحرار. كلهم مذعنون لعظمته, خاضعون لعزته, منتظرون لمجازاته, راجون ثوابه, خائفون من عقابه, فلذلك خصه بالذكر, وإلا, فهو المالك ليوم الدين ولغيره من الأيام.
\end{Arabic}
\end{taggedblock}
\begin{taggedblock}{tafseer_sadi_translation}
{ مَالِكِ يَوْمِ الدِّينِ }
المالك: هو من اتصف بصفة الملك التي من آثارها أنه يأمر وينهى, ويثيب ويعاقب, ويتصرف بمماليكه بجميع أنواع التصرفات, وأضاف الملك ليوم الدين, وهو يوم القيامة, يوم يدان الناس فيه بأعمالهم, خيرها وشرها, لأن في ذلك اليوم, يظهر للخلق تمام الظهور, كمال ملكه وعدله وحكمته, وانقطاع أملاك الخلائق. حتى
[إنه]
يستوي في ذلك اليوم, الملوك والرعايا والعبيد والأحرار. كلهم مذعنون لعظمته, خاضعون لعزته, منتظرون لمجازاته, راجون ثوابه, خائفون من عقابه, فلذلك خصه بالذكر, وإلا, فهو المالك ليوم الدين ولغيره من الأيام.
\end{taggedblock}
\begin{taggedblock}{license_attribution_aayah}
\input{license-attribution}
\end{taggedblock}
\begin{comment}
Please use the following for footnotes:- Sample\footnoteQ{Text of Qur'an footnote goes here.}.
Sample\footnoteT{Text of Tafseer footnote goes here.}.
\end{comment}