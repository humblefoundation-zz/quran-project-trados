%This file is used by a ruby script as a template for each aayah
\begin{comment}
The following strings are to be replaced by a script, in order to use this file as a template (all upper case):-
* sN = sūrah number, without leading zeros
* aYN = āyah number, without leading zeros
* aRABIC_TEXT_TASHKEEL = the text of the aayah, with tashkeel marks
* aRABIC_TEXT_WITHOUT_TASHKEEL = the text of the aayah, without tashkeel marks
* tAFSEER_SADI_ARABIC = the tafseer of the aayah from as-sa'di
\end{comment}
\begin{comment}
The following tags are declared here:-
quran_com_link
quran_com_arabic_image
arabic_text_tashkeel
our_translation
variations_in_reading
translator_comments
aayah_tags
tafseer_sadi_arabic
tafseer_sadi_translation
\end{comment}
\begin{taggedblock}{quran_com_link}
\href{http://quran.com/49/5}{Link to 49:5 on quran.com}
\end{taggedblock}
\begin{taggedblock}{quran_com_arabic_image}
\includegraphics{49_5}
\end{taggedblock}
\begin{taggedblock}{arabic_text_tashkeel}
\begin{Arabic}
وَلَوْ أَنَّهُمْ صَبَرُوا حَتَّىٰ تَخْرُجَ إِلَيْهِمْ لَكَانَ خَيْرًا لَهُمْ وَاللَّهُ غَفُورٌ رَحِيمٌ
\end{Arabic}
\end{taggedblock}
\begin{taggedblock}{our_translation}
ولو أنهم صبروا حتى تخرج إليهم لكان خيرا لهم والله غفور رحيم
\end{taggedblock}
\begin{taggedblock}{variations_in_reading}
%This section is optional, for translating different wordings. For each different wording, paste the translation again, with the changes from Hafṣ highlighted in bold.
\end{taggedblock}
\begin{taggedblock}{translator_comments}
%Put any comments that you have as a translator, including issues of concern, or major decisions that you made when translating.
\end{taggedblock}
\begin{taggedblock}{aayah_tags}
%Put tags here separated by commas, e.g.: tawheed,prophets,yusuf,dua
\end{taggedblock}
\begin{taggedblock}{tafseer_sadi_arabic}
\begin{Arabic}
فأدب العبد، عنوان عقله، وأن الله مريد به الخير، ولهذا قال:
{ وَلَوْ أَنَّهُمْ صَبَرُوا حَتَّى تَخْرُجَ إِلَيْهِمْ لَكَانَ خَيْرًا لَهُمْ وَاللَّهُ غَفُورٌ رَحِيمٌ }
أي: غفور لما صدر عن عباده من الذنوب، والإخلال بالآداب، رحيم بهم، حيث لم يعاجلهم بذنوبهم بالعقوبات والمثلات.
\end{Arabic}
\end{taggedblock}
\begin{taggedblock}{tafseer_sadi_translation}
فأدب العبد، عنوان عقله، وأن الله مريد به الخير، ولهذا قال:
{ وَلَوْ أَنَّهُمْ صَبَرُوا حَتَّى تَخْرُجَ إِلَيْهِمْ لَكَانَ خَيْرًا لَهُمْ وَاللَّهُ غَفُورٌ رَحِيمٌ }
أي: غفور لما صدر عن عباده من الذنوب، والإخلال بالآداب، رحيم بهم، حيث لم يعاجلهم بذنوبهم بالعقوبات والمثلات.
\end{taggedblock}
\input{license-attribution}
\begin{comment}
Please use the following for footnotes:- Sample\footnoteQ{Text of Qur'an footnote goes here.}.
Sample\footnoteT{Text of Tafseer footnote goes here.}.
\end{comment}