%This file is used by a ruby script as a template for each aayah
\begin{comment}
The following strings are to be replaced by a script, in order to use this file as a template (all upper case):-
* sN = sūrah number, without leading zeros
* aYN = āyah number, without leading zeros
* aRABIC_TEXT_TASHKEEL = the text of the aayah, with tashkeel marks
* aRABIC_TEXT_WITHOUT_TASHKEEL = the text of the aayah, without tashkeel marks
* tAFSEER_SADI_ARABIC = the tafseer of the aayah from as-sa'di
\end{comment}
\begin{comment}
The following tags are declared here:-
quran_com_link
quran_com_arabic_image
arabic_text_tashkeel
our_translation
variations_in_reading
translator_comments
aayah_tags
tafseer_sadi_arabic
tafseer_sadi_translation
license_attribution_aayah
\end{comment}
\begin{taggedblock}{quran_com_link}
\href{http://quran.com/49/1}{Link to 49:1 on quran.com}
\end{taggedblock}
\begin{taggedblock}{quran_com_arabic_image}
\includegraphics{49_1}
\end{taggedblock}
\begin{taggedblock}{arabic_text_tashkeel}
\begin{Arabic}
يَا أَيُّهَا الَّذِينَ آمَنُوا لَا تُقَدِّمُوا بَيْنَ يَدَيِ اللَّهِ وَرَسُولِهِ وَاتَّقُوا اللَّهَ إِنَّ اللَّهَ سَمِيعٌ عَلِيمٌ
\end{Arabic}
\end{taggedblock}
\begin{taggedblock}{our_translation}
يا أيها الذين آمنوا لا تقدموا بين يدي الله ورسوله واتقوا الله إن الله سميع عليم
\end{taggedblock}
\begin{taggedblock}{variations_in_reading}
%This section is optional, for translating different wordings. For each different wording, paste the translation again, with the changes from Hafṣ highlighted in bold.
\end{taggedblock}
\begin{taggedblock}{translator_comments}
%Put any comments that you have as a translator, including issues of concern, or major decisions that you made when translating.
\end{taggedblock}
\begin{taggedblock}{aayah_tags}
%Put tags here separated by commas, e.g.: tawheed,prophets,yusuf,dua
\end{taggedblock}
\begin{taggedblock}{tafseer_sadi_arabic}
\begin{Arabic}
هذا متضمن للأدب، مع الله تعالى، ومع رسول الله صلى الله عليه وسلم، والتعظيم له ، واحترامه، وإكرامه، فأمر
[الله]
عباده المؤمنين، بما يقتضيه الإيمان، بالله وبرسوله، من امتثال أوامر الله، واجتناب نواهيه، وأن يكونوا ماشين، خلف أوامر الله، متبعين لسنة رسول الله صلى الله عليه وسلم، في جميع أمورهم، و
[أن]
لا يتقدموا بين يدي الله ورسوله، ولا يقولوا، حتى يقول، ولا يأمروا، حتى يأمر، فإن هذا، حقيقة الأدب الواجب، مع الله ورسوله، وهو عنوان سعادة العبد وفلاحه، وبفواته، تفوته السعادة الأبدية، والنعيم السرمدي، وفي هذا، النهي
[الشديد]
عن تقديم قول غير الرسول صلى الله عليه وسلم، على قوله، فإنه متى استبانت سنة رسول الله صلى الله عليه وسلم، وجب اتباعها، وتقديمها على غيرها، كائنا ما كان

ثم أمر الله بتقواه عمومًا، وهي كما قال طلق بن حبيب: أن تعمل بطاعة الله، على نور من الله، ترجو ثواب الله، وأن تترك معصية الله، على نور من الله، تخشى عقاب الله.

وقوله:
{ إِنَّ اللَّهَ سَمِيعٌ }
أي: لجميع الأصوات في جميع الأوقات، في خفي المواضع والجهات،
{ عَلِيمٌ }
بالظواهر والبواطن، والسوابق واللواحق، والواجبات والمستحيلات والممكنات

وفي ذكر الاسمين الكريمين -بعد النهي عن التقدم بين يدي الله ورسوله، والأمر بتقواه- حث على امتثال تلك الأوامر الحسنة، والآداب المستحسنة، وترهيب عن عدم الامتثال
\end{Arabic}
\end{taggedblock}
\begin{taggedblock}{tafseer_sadi_translation}
هذا متضمن للأدب، مع الله تعالى، ومع رسول الله صلى الله عليه وسلم، والتعظيم له ، واحترامه، وإكرامه، فأمر
[الله]
عباده المؤمنين، بما يقتضيه الإيمان، بالله وبرسوله، من امتثال أوامر الله، واجتناب نواهيه، وأن يكونوا ماشين، خلف أوامر الله، متبعين لسنة رسول الله صلى الله عليه وسلم، في جميع أمورهم، و
[أن]
لا يتقدموا بين يدي الله ورسوله، ولا يقولوا، حتى يقول، ولا يأمروا، حتى يأمر، فإن هذا، حقيقة الأدب الواجب، مع الله ورسوله، وهو عنوان سعادة العبد وفلاحه، وبفواته، تفوته السعادة الأبدية، والنعيم السرمدي، وفي هذا، النهي
[الشديد]
عن تقديم قول غير الرسول صلى الله عليه وسلم، على قوله، فإنه متى استبانت سنة رسول الله صلى الله عليه وسلم، وجب اتباعها، وتقديمها على غيرها، كائنا ما كان

ثم أمر الله بتقواه عمومًا، وهي كما قال طلق بن حبيب: أن تعمل بطاعة الله، على نور من الله، ترجو ثواب الله، وأن تترك معصية الله، على نور من الله، تخشى عقاب الله.

وقوله:
{ إِنَّ اللَّهَ سَمِيعٌ }
أي: لجميع الأصوات في جميع الأوقات، في خفي المواضع والجهات،
{ عَلِيمٌ }
بالظواهر والبواطن، والسوابق واللواحق، والواجبات والمستحيلات والممكنات

وفي ذكر الاسمين الكريمين -بعد النهي عن التقدم بين يدي الله ورسوله، والأمر بتقواه- حث على امتثال تلك الأوامر الحسنة، والآداب المستحسنة، وترهيب عن عدم الامتثال
\end{taggedblock}
\begin{taggedblock}{license_attribution_aayah}
\input{license-attribution}
\end{taggedblock}
\begin{comment}
Please use the following for footnotes:- Sample\footnoteQ{Text of Qur'an footnote goes here.}.
Sample\footnoteT{Text of Tafseer footnote goes here.}.
\end{comment}