%This file is used by a ruby script as a template for each aayah
\begin{comment}
The following strings are to be replaced by a script, in order to use this file as a template (all upper case):-
* sN = sūrah number, without leading zeros
* aYN = āyah number, without leading zeros
* aRABIC_TEXT_TASHKEEL = the text of the aayah, with tashkeel marks
* aRABIC_TEXT_WITHOUT_TASHKEEL = the text of the aayah, without tashkeel marks
* tAFSEER_SADI_ARABIC = the tafseer of the aayah from as-sa'di
\end{comment}
\begin{comment}
The following tags are declared here:-
quran_com_link
quran_com_arabic_image
arabic_text_tashkeel
our_translation
variations_in_reading
translator_comments
aayah_tags
tafseer_sadi_arabic
tafseer_sadi_translation
license_attribution_aayah
\end{comment}
\begin{taggedblock}{quran_com_link}
\href{http://quran.com/49/17}{Link to 49:17 on quran.com}
\end{taggedblock}
\begin{taggedblock}{quran_com_arabic_image}
\includegraphics{49_17}
\end{taggedblock}
\begin{taggedblock}{arabic_text_tashkeel}
\begin{Arabic}
يَمُنُّونَ عَلَيْكَ أَنْ أَسْلَمُوا قُلْ لَا تَمُنُّوا عَلَيَّ إِسْلَامَكُمْ بَلِ اللَّهُ يَمُنُّ عَلَيْكُمْ أَنْ هَدَاكُمْ لِلْإِيمَانِ إِنْ كُنْتُمْ صَادِقِينَ
\end{Arabic}
\end{taggedblock}
\begin{taggedblock}{our_translation}
يمنون عليك أن أسلموا قل لا تمنوا علي إسلامكم بل الله يمن عليكم أن هداكم للإيمان إن كنتم صادقين
\end{taggedblock}
\begin{taggedblock}{variations_in_reading}
%This section is optional, for translating different wordings. For each different wording, paste the translation again, with the changes from Hafṣ highlighted in bold.
\end{taggedblock}
\begin{taggedblock}{translator_comments}
%Put any comments that you have as a translator, including issues of concern, or major decisions that you made when translating.
\end{taggedblock}
\begin{taggedblock}{aayah_tags}
%Put tags here separated by commas, e.g.: tawheed,prophets,yusuf,dua
\end{taggedblock}
\begin{taggedblock}{tafseer_sadi_arabic}
\begin{Arabic}
هذه حالة من أحوال من ادعى لنفسه الإيمان، وليس به، فإنه إما أن يكون ذلك تعليمًا لله، وقد علم أنه عالم بكل شيء، وإما أن يكون قصدهم بهذا الكلام، المنة على رسوله، وأنهم قد بذلوا له
[وتبرعوا]
بما ليس من مصالحهم، بل هو من حظوظه الدنيوية، وهذا تجمل بما لا يجمل، وفخر بما لا ينبغي لهم أن يفتخروا على رسوله به  فإن المنة لله تعالى عليهم، فكما أنه تعالى يمن  عليهم، بالخلق والرزق، والنعم الظاهرة والباطنة، فمنته عليهم بهدايتهم إلى الإسلام، ومنته عليهم بالإيمان، أعظم  من كل شيء، ولهذا قال تعالى:
{ يَمُنُّونَ عَلَيْكَ أَنْ أَسْلَمُوا قُلْ لَا تَمُنُّوا عَلَيَّ إِسْلَامَكُمْ بَلِ اللَّهُ يَمُنُّ عَلَيْكُمْ أَنْ هَدَاكُمْ لِلْإِيمَانِ إِنْ كُنْتُمْ صَادِقِينَ }
.
\end{Arabic}
\end{taggedblock}
\begin{taggedblock}{tafseer_sadi_translation}
هذه حالة من أحوال من ادعى لنفسه الإيمان، وليس به، فإنه إما أن يكون ذلك تعليمًا لله، وقد علم أنه عالم بكل شيء، وإما أن يكون قصدهم بهذا الكلام، المنة على رسوله، وأنهم قد بذلوا له
[وتبرعوا]
بما ليس من مصالحهم، بل هو من حظوظه الدنيوية، وهذا تجمل بما لا يجمل، وفخر بما لا ينبغي لهم أن يفتخروا على رسوله به  فإن المنة لله تعالى عليهم، فكما أنه تعالى يمن  عليهم، بالخلق والرزق، والنعم الظاهرة والباطنة، فمنته عليهم بهدايتهم إلى الإسلام، ومنته عليهم بالإيمان، أعظم  من كل شيء، ولهذا قال تعالى:
{ يَمُنُّونَ عَلَيْكَ أَنْ أَسْلَمُوا قُلْ لَا تَمُنُّوا عَلَيَّ إِسْلَامَكُمْ بَلِ اللَّهُ يَمُنُّ عَلَيْكُمْ أَنْ هَدَاكُمْ لِلْإِيمَانِ إِنْ كُنْتُمْ صَادِقِينَ }
.
\end{taggedblock}
\begin{taggedblock}{license_attribution_aayah}
\input{license-attribution}
\end{taggedblock}
\begin{comment}
Please use the following for footnotes:- Sample\footnoteQ{Text of Qur'an footnote goes here.}.
Sample\footnoteT{Text of Tafseer footnote goes here.}.
\end{comment}