%This file is used by a ruby script as a template for each aayah
\begin{comment}
The following strings are to be replaced by a script, in order to use this file as a template (all upper case):-
* sN = sūrah number, without leading zeros
* aYN = āyah number, without leading zeros
* aRABIC_TEXT_TASHKEEL = the text of the aayah, with tashkeel marks
* aRABIC_TEXT_WITHOUT_TASHKEEL = the text of the aayah, without tashkeel marks
* tAFSEER_SADI_ARABIC = the tafseer of the aayah from as-sa'di
\end{comment}
\begin{comment}
The following tags are declared here:-
quran_com_link
quran_com_arabic_image
arabic_text_tashkeel
our_translation
variations_in_reading
translator_comments
aayah_tags
tafseer_sadi_arabic
tafseer_sadi_translation
license_attribution_aayah
\end{comment}
\begin{taggedblock}{quran_com_link}
\href{http://quran.com/49/13}{Link to 49:13 on quran.com}
\end{taggedblock}
\begin{taggedblock}{quran_com_arabic_image}
\includegraphics{49_13}
\end{taggedblock}
\begin{taggedblock}{arabic_text_tashkeel}
\begin{Arabic}
يَا أَيُّهَا النَّاسُ إِنَّا خَلَقْنَاكُمْ مِنْ ذَكَرٍ وَأُنْثَىٰ وَجَعَلْنَاكُمْ شُعُوبًا وَقَبَائِلَ لِتَعَارَفُوا إِنَّ أَكْرَمَكُمْ عِنْدَ اللَّهِ أَتْقَاكُمْ إِنَّ اللَّهَ عَلِيمٌ خَبِيرٌ
\end{Arabic}
\end{taggedblock}
\begin{taggedblock}{our_translation}
يا أيها الناس إنا خلقناكم من ذكر وأنثى وجعلناكم شعوبا وقبائل لتعارفوا إن أكرمكم عند الله أتقاكم إن الله عليم خبير
\end{taggedblock}
\begin{taggedblock}{variations_in_reading}
%This section is optional, for translating different wordings. For each different wording, paste the translation again, with the changes from Hafṣ highlighted in bold.
\end{taggedblock}
\begin{taggedblock}{translator_comments}
%Put any comments that you have as a translator, including issues of concern, or major decisions that you made when translating.
\end{taggedblock}
\begin{taggedblock}{aayah_tags}
%Put tags here separated by commas, e.g.: tawheed,prophets,yusuf,dua
\end{taggedblock}
\begin{taggedblock}{tafseer_sadi_arabic}
\begin{Arabic}
يخبر تعالى أنه خلق بني آدم، من أصل واحد، وجنس واحد، وكلهم من ذكر وأنثى، ويرجعون جميعهم إلى آدم وحواء، ولكن الله
[تعالى]
بث منهما رجالاً كثيرا ونساء، وفرقهم، وجعلهم شعوبًا وقبائل أي: قبائل صغارًا وكبارًا، وذلك لأجل أن يتعارفوا، فإنهم لو استقل كل واحد منهم بنفسه، لم يحصل بذلك، التعارف الذي يترتب عليه التناصر والتعاون، والتوارث، والقيام بحقوق الأقارب، ولكن الله جعلهم شعوبًا وقبائل، لأجل أن تحصل هذه الأمور وغيرها، مما يتوقف على التعارف، ولحوق الأنساب، ولكن الكرم بالتقوى، فأكرمهم عند الله، أتقاهم، وهو أكثرهم طاعة وانكفافًا عن المعاصي، لا أكثرهم قرابة وقومًا، ولا أشرفهم نسبًا، ولكن الله تعالى عليم خبير، يعلم من يقوم منهم بتقوى الله، ظاهرًا وباطنًا، ممن يقوم بذلك، ظاهرًا لا باطنًا، فيجازي كلا، بما يستحق.

وفي هذه الآية دليل على أن معرفة الأنساب، مطلوبة مشروعة، لأن الله جعلهم شعوبًا وقبائل، لأجل ذلك.
\end{Arabic}
\end{taggedblock}
\begin{taggedblock}{tafseer_sadi_translation}
يخبر تعالى أنه خلق بني آدم، من أصل واحد، وجنس واحد، وكلهم من ذكر وأنثى، ويرجعون جميعهم إلى آدم وحواء، ولكن الله
[تعالى]
بث منهما رجالاً كثيرا ونساء، وفرقهم، وجعلهم شعوبًا وقبائل أي: قبائل صغارًا وكبارًا، وذلك لأجل أن يتعارفوا، فإنهم لو استقل كل واحد منهم بنفسه، لم يحصل بذلك، التعارف الذي يترتب عليه التناصر والتعاون، والتوارث، والقيام بحقوق الأقارب، ولكن الله جعلهم شعوبًا وقبائل، لأجل أن تحصل هذه الأمور وغيرها، مما يتوقف على التعارف، ولحوق الأنساب، ولكن الكرم بالتقوى، فأكرمهم عند الله، أتقاهم، وهو أكثرهم طاعة وانكفافًا عن المعاصي، لا أكثرهم قرابة وقومًا، ولا أشرفهم نسبًا، ولكن الله تعالى عليم خبير، يعلم من يقوم منهم بتقوى الله، ظاهرًا وباطنًا، ممن يقوم بذلك، ظاهرًا لا باطنًا، فيجازي كلا، بما يستحق.

وفي هذه الآية دليل على أن معرفة الأنساب، مطلوبة مشروعة، لأن الله جعلهم شعوبًا وقبائل، لأجل ذلك.
\end{taggedblock}
\begin{taggedblock}{license_attribution_aayah}
\input{license-attribution}
\end{taggedblock}
\begin{comment}
Please use the following for footnotes:- Sample\footnoteQ{Text of Qur'an footnote goes here.}.
Sample\footnoteT{Text of Tafseer footnote goes here.}.
\end{comment}