%This file is used by a ruby script as a template for each aayah
\begin{comment}
The following strings are to be replaced by a script, in order to use this file as a template (all upper case):-
* sN = sūrah number, without leading zeros
* aYN = āyah number, without leading zeros
* aRABIC_TEXT_TASHKEEL = the text of the aayah, with tashkeel marks
* aRABIC_TEXT_WITHOUT_TASHKEEL = the text of the aayah, without tashkeel marks
* tAFSEER_SADI_ARABIC = the tafseer of the aayah from as-sa'di
\end{comment}
\begin{comment}
The following tags are declared here:-
quran_com_link
quran_com_arabic_image
arabic_text_tashkeel
our_translation
variations_in_reading
translator_comments
aayah_tags
tafseer_sadi_arabic
tafseer_sadi_translation
license_attribution_aayah
\end{comment}
\begin{taggedblock}{quran_com_link}
\href{http://quran.com/49/14}{Link to 49:14 on quran.com}
\end{taggedblock}
\begin{taggedblock}{quran_com_arabic_image}
\includegraphics{49_14}
\end{taggedblock}
\begin{taggedblock}{arabic_text_tashkeel}
\begin{Arabic}
قَالَتِ الْأَعْرَابُ آمَنَّا قُلْ لَمْ تُؤْمِنُوا وَلَٰكِنْ قُولُوا أَسْلَمْنَا وَلَمَّا يَدْخُلِ الْإِيمَانُ فِي قُلُوبِكُمْ وَإِنْ تُطِيعُوا اللَّهَ وَرَسُولَهُ لَا يَلِتْكُمْ مِنْ أَعْمَالِكُمْ شَيْئًا إِنَّ اللَّهَ غَفُورٌ رَحِيمٌ
\end{Arabic}
\end{taggedblock}
\begin{taggedblock}{our_translation}
قالت الأعراب آمنا قل لم تؤمنوا ولكن قولوا أسلمنا ولما يدخل الإيمان في قلوبكم وإن تطيعوا الله ورسوله لا يلتكم من أعمالكم شيئا إن الله غفور رحيم
\end{taggedblock}
\begin{taggedblock}{variations_in_reading}
%This section is optional, for translating different wordings. For each different wording, paste the translation again, with the changes from Hafṣ highlighted in bold.
\end{taggedblock}
\begin{taggedblock}{translator_comments}
%Put any comments that you have as a translator, including issues of concern, or major decisions that you made when translating.
\end{taggedblock}
\begin{taggedblock}{aayah_tags}
%Put tags here separated by commas, e.g.: tawheed,prophets,yusuf,dua
\end{taggedblock}
\begin{taggedblock}{tafseer_sadi_arabic}
\begin{Arabic}
يخبر تعالى عن مقالة الأعراب، الذين دخلوا في الإسلام في عهد رسول الله صلى الله عليه وسلم، دخولاً من غير بصيرة، ولا قيام بما يجب ويقتضيه الإيمان، أنهم ادعوا مع هذا وقالوا: آمنا أي: إيمانًا كاملاً، مستوفيًا لجميع أموره هذا موجب هذا الكلام، فأمر الله رسوله، أن يرد عليهم، فقال:
{ قُلْ لَمْ تُؤْمِنُوا }
أي: لا تدعوا لأنفسكم مقام الإيمان، ظاهرًا، وباطنًا، كاملاً.

{ وَلَكِنْ قُولُوا أَسْلَمْنَا }
أي: دخلنا في الإسلام، واقتصروا على ذلك.

{ و }
السبب في ذلك، أنه
{ لَمَّا يَدْخُلِ الْإِيمَانُ فِي قُلُوبِكُمْ }
وإنما آمنتم خوفًا، أو رجاء، أو نحو ذلك، مما هو السبب في إيمانكم، فلذلك لم تدخل بشاشة الإيمان في قلوبكم، وفي قوله:
{ وَلَمَّا يَدْخُلِ الْإِيمَانُ فِي قُلُوبِكُمْ }
أي: وقت هذا الكلام، الذي صدر منكم فكان فيه إشارة إلى أحوالهم بعد ذلك، فإن كثيرًا منهم، من الله عليهم بالإيمان الحقيقي، والجهاد في سبيل الله،
{ وَإِنْ تُطِيعُوا اللَّهَ وَرَسُولَهُ }
بفعل خير، أو ترك شر
{ لَا يَلِتْكُمْ مِنْ أَعْمَالِكُمْ شَيْئًا }
أي: لا ينقصكم منها، مثقال ذرة، بل يوفيكم إياها، أكمل ما تكون لا تفقدون منها، صغيرًا، ولا كبيرًا،
{ إِنَّ اللَّهَ غَفُورٌ رَحِيمٌ }
أي: غفور لمن تاب إليه وأناب، رحيم به، حيث قبل توبته.
\end{Arabic}
\end{taggedblock}
\begin{taggedblock}{tafseer_sadi_translation}
يخبر تعالى عن مقالة الأعراب، الذين دخلوا في الإسلام في عهد رسول الله صلى الله عليه وسلم، دخولاً من غير بصيرة، ولا قيام بما يجب ويقتضيه الإيمان، أنهم ادعوا مع هذا وقالوا: آمنا أي: إيمانًا كاملاً، مستوفيًا لجميع أموره هذا موجب هذا الكلام، فأمر الله رسوله، أن يرد عليهم، فقال:
{ قُلْ لَمْ تُؤْمِنُوا }
أي: لا تدعوا لأنفسكم مقام الإيمان، ظاهرًا، وباطنًا، كاملاً.

{ وَلَكِنْ قُولُوا أَسْلَمْنَا }
أي: دخلنا في الإسلام، واقتصروا على ذلك.

{ و }
السبب في ذلك، أنه
{ لَمَّا يَدْخُلِ الْإِيمَانُ فِي قُلُوبِكُمْ }
وإنما آمنتم خوفًا، أو رجاء، أو نحو ذلك، مما هو السبب في إيمانكم، فلذلك لم تدخل بشاشة الإيمان في قلوبكم، وفي قوله:
{ وَلَمَّا يَدْخُلِ الْإِيمَانُ فِي قُلُوبِكُمْ }
أي: وقت هذا الكلام، الذي صدر منكم فكان فيه إشارة إلى أحوالهم بعد ذلك، فإن كثيرًا منهم، من الله عليهم بالإيمان الحقيقي، والجهاد في سبيل الله،
{ وَإِنْ تُطِيعُوا اللَّهَ وَرَسُولَهُ }
بفعل خير، أو ترك شر
{ لَا يَلِتْكُمْ مِنْ أَعْمَالِكُمْ شَيْئًا }
أي: لا ينقصكم منها، مثقال ذرة، بل يوفيكم إياها، أكمل ما تكون لا تفقدون منها، صغيرًا، ولا كبيرًا،
{ إِنَّ اللَّهَ غَفُورٌ رَحِيمٌ }
أي: غفور لمن تاب إليه وأناب، رحيم به، حيث قبل توبته.
\end{taggedblock}
\begin{taggedblock}{license_attribution_aayah}
\input{license-attribution}
\end{taggedblock}
\begin{comment}
Please use the following for footnotes:- Sample\footnoteQ{Text of Qur'an footnote goes here.}.
Sample\footnoteT{Text of Tafseer footnote goes here.}.
\end{comment}