%This file is used by a ruby script as a template for each aayah
\begin{comment}
The following strings are to be replaced by a script, in order to use this file as a template (all upper case):-
* sN = sūrah number, without leading zeros
* aYN = āyah number, without leading zeros
* aRABIC_TEXT_TASHKEEL = the text of the aayah, with tashkeel marks
* aRABIC_TEXT_WITHOUT_TASHKEEL = the text of the aayah, without tashkeel marks
* tAFSEER_SADI_ARABIC = the tafseer of the aayah from as-sa'di
\end{comment}
\begin{comment}
The following tags are declared here:-
quran_com_link
quran_com_arabic_image
arabic_text_tashkeel
our_translation
variations_in_reading
translator_comments
aayah_tags
tafseer_sadi_arabic
tafseer_sadi_translation
license_attribution_aayah
\end{comment}
\begin{taggedblock}{quran_com_link}
\href{http://quran.com/49/11}{Link to 49:11 on quran.com}
\end{taggedblock}
\begin{taggedblock}{quran_com_arabic_image}
\includegraphics{49_11}
\end{taggedblock}
\begin{taggedblock}{arabic_text_tashkeel}
\begin{Arabic}
يَا أَيُّهَا الَّذِينَ آمَنُوا لَا يَسْخَرْ قَوْمٌ مِنْ قَوْمٍ عَسَىٰ أَنْ يَكُونُوا خَيْرًا مِنْهُمْ وَلَا نِسَاءٌ مِنْ نِسَاءٍ عَسَىٰ أَنْ يَكُنَّ خَيْرًا مِنْهُنَّ وَلَا تَلْمِزُوا أَنْفُسَكُمْ وَلَا تَنَابَزُوا بِالْأَلْقَابِ بِئْسَ الِاسْمُ الْفُسُوقُ بَعْدَ الْإِيمَانِ وَمَنْ لَمْ يَتُبْ فَأُولَٰئِكَ هُمُ الظَّالِمُونَ
\end{Arabic}
\end{taggedblock}
\begin{taggedblock}{our_translation}
يا أيها الذين آمنوا لا يسخر قوم من قوم عسى أن يكونوا خيرا منهم ولا نساء من نساء عسى أن يكن خيرا منهن ولا تلمزوا أنفسكم ولا تنابزوا بالألقاب بئس الاسم الفسوق بعد الإيمان ومن لم يتب فأولئك هم الظالمون
\end{taggedblock}
\begin{taggedblock}{variations_in_reading}
%This section is optional, for translating different wordings. For each different wording, paste the translation again, with the changes from Hafṣ highlighted in bold.
\end{taggedblock}
\begin{taggedblock}{translator_comments}
%Put any comments that you have as a translator, including issues of concern, or major decisions that you made when translating.
\end{taggedblock}
\begin{taggedblock}{aayah_tags}
%Put tags here separated by commas, e.g.: tawheed,prophets,yusuf,dua
\end{taggedblock}
\begin{taggedblock}{tafseer_sadi_arabic}
\begin{Arabic}
وهذا أيضًا، من حقوق المؤمنين، بعضهم على بعض، أن
{ لَا يَسْخَرْ قَومٌ مِنْ قَوْمٍ }
بكل كلام، وقول، وفعل دال على تحقير الأخ المسلم، فإن ذلك حرام، لا يجوز، وهو دال على إعجاب الساخر بنفسه، وعسى أن يكون المسخور به خيرًا من الساخر، كما هو  الغالب والواقع، فإن السخرية، لا تقع إلا من قلب ممتلئ من مساوئ الأخلاق، متحل بكل خلق ذميم، ولهذا قال النبي صلى الله عليه وسلم
"بحسب امرئ من الشر، أن يحقر أخاه المسلم"

ثم قال:
{ وَلَا تَلْمِزُوا أَنْفُسَكُمْ }
أي: لا يعب بعضكم على بعض، واللمز: بالقول، والهمز: بالفعل، وكلاهما منهي عنه حرام، متوعد عليه بالنار.

كما قال تعالى:
{ وَيْلٌ لِكُلِّ هُمَزَةٍ لُمَزَةٍ }
الآية، وسمي الأخ المؤمن  نفسًا لأخيه، لأن المؤمنين ينبغي أن يكون هكذا حالهم كالجسد الواحد، ولأنه إذا همز غيره، أوجب للغير أن يهمزه، فيكون هو المتسبب لذلك.

{ وَلَا تَنَابَزُوا بِالْأَلْقَابِ }
أي: لا يعير أحدكم أخاه، ويلقبه بلقب ذم يكره أن يطلق عليه  وهذا هو التنابز، وأما الألقاب غير المذمومة، فلا تدخل في هذا.

{ بِئْسَ الِاسْمُ الْفُسُوقُ بَعْدَ الْإِيمَانِ }
أي: بئسما تبدلتم عن الإيمان والعمل بشرائعه، وما تقتضيه، بالإعراض عن أوامره ونواهيه، باسم الفسوق والعصيان، الذي هو التنابز بالألقاب.

{ وَمَنْ لَمْ يَتُبْ فَأُولَئِكَ هُمُ الظَّالِمُونَ }
فهذا
[هو]
الواجب على العبد، أن يتوب إلى الله تعالى، ويخرج من حق أخيه المسلم، باستحلاله، والاستغفار، والمدح له مقابلة
[على]
ذمه.

{ وَمَنْ لَمْ يَتُبْ فَأُولَئِكَ هُمُ الظَّالِمُونَ }
فالناس قسمان: ظالم لنفسه غير تائب، وتائب مفلح، ولا ثم قسم ثالث غيرهما.
\end{Arabic}
\end{taggedblock}
\begin{taggedblock}{tafseer_sadi_translation}
وهذا أيضًا، من حقوق المؤمنين، بعضهم على بعض، أن
{ لَا يَسْخَرْ قَومٌ مِنْ قَوْمٍ }
بكل كلام، وقول، وفعل دال على تحقير الأخ المسلم، فإن ذلك حرام، لا يجوز، وهو دال على إعجاب الساخر بنفسه، وعسى أن يكون المسخور به خيرًا من الساخر، كما هو  الغالب والواقع، فإن السخرية، لا تقع إلا من قلب ممتلئ من مساوئ الأخلاق، متحل بكل خلق ذميم، ولهذا قال النبي صلى الله عليه وسلم
"بحسب امرئ من الشر، أن يحقر أخاه المسلم"

ثم قال:
{ وَلَا تَلْمِزُوا أَنْفُسَكُمْ }
أي: لا يعب بعضكم على بعض، واللمز: بالقول، والهمز: بالفعل، وكلاهما منهي عنه حرام، متوعد عليه بالنار.

كما قال تعالى:
{ وَيْلٌ لِكُلِّ هُمَزَةٍ لُمَزَةٍ }
الآية، وسمي الأخ المؤمن  نفسًا لأخيه، لأن المؤمنين ينبغي أن يكون هكذا حالهم كالجسد الواحد، ولأنه إذا همز غيره، أوجب للغير أن يهمزه، فيكون هو المتسبب لذلك.

{ وَلَا تَنَابَزُوا بِالْأَلْقَابِ }
أي: لا يعير أحدكم أخاه، ويلقبه بلقب ذم يكره أن يطلق عليه  وهذا هو التنابز، وأما الألقاب غير المذمومة، فلا تدخل في هذا.

{ بِئْسَ الِاسْمُ الْفُسُوقُ بَعْدَ الْإِيمَانِ }
أي: بئسما تبدلتم عن الإيمان والعمل بشرائعه، وما تقتضيه، بالإعراض عن أوامره ونواهيه، باسم الفسوق والعصيان، الذي هو التنابز بالألقاب.

{ وَمَنْ لَمْ يَتُبْ فَأُولَئِكَ هُمُ الظَّالِمُونَ }
فهذا
[هو]
الواجب على العبد، أن يتوب إلى الله تعالى، ويخرج من حق أخيه المسلم، باستحلاله، والاستغفار، والمدح له مقابلة
[على]
ذمه.

{ وَمَنْ لَمْ يَتُبْ فَأُولَئِكَ هُمُ الظَّالِمُونَ }
فالناس قسمان: ظالم لنفسه غير تائب، وتائب مفلح، ولا ثم قسم ثالث غيرهما.
\end{taggedblock}
\begin{taggedblock}{license_attribution_aayah}
\input{license-attribution}
\end{taggedblock}
\begin{comment}
Please use the following for footnotes:- Sample\footnoteQ{Text of Qur'an footnote goes here.}.
Sample\footnoteT{Text of Tafseer footnote goes here.}.
\end{comment}