%This file is used by a ruby script as a template for each aayah
\begin{comment}
The following strings are to be replaced by a script, in order to use this file as a template (all upper case):-
* sN = sūrah number, without leading zeros
* aYN = āyah number, without leading zeros
* aRABIC_TEXT_TASHKEEL = the text of the aayah, with tashkeel marks
* aRABIC_TEXT_WITHOUT_TASHKEEL = the text of the aayah, without tashkeel marks
* tAFSEER_SADI_ARABIC = the tafseer of the aayah from as-sa'di
\end{comment}
\begin{comment}
The following tags are declared here:-
quran_com_link
quran_com_arabic_image
arabic_text_tashkeel
our_translation
variations_in_reading
translator_comments
aayah_tags
tafseer_sadi_arabic
tafseer_sadi_translation
license_attribution_aayah
\end{comment}
\begin{taggedblock}{quran_com_link}
\href{http://quran.com/49/2}{Link to 49:2 on quran.com}
\end{taggedblock}
\begin{taggedblock}{quran_com_arabic_image}
\includegraphics{49_2}
\end{taggedblock}
\begin{taggedblock}{arabic_text_tashkeel}
\begin{Arabic}
يَا أَيُّهَا الَّذِينَ آمَنُوا لَا تَرْفَعُوا أَصْوَاتَكُمْ فَوْقَ صَوْتِ النَّبِيِّ وَلَا تَجْهَرُوا لَهُ بِالْقَوْلِ كَجَهْرِ بَعْضِكُمْ لِبَعْضٍ أَنْ تَحْبَطَ أَعْمَالُكُمْ وَأَنْتُمْ لَا تَشْعُرُونَ
\end{Arabic}
\end{taggedblock}
\begin{taggedblock}{our_translation}
يا أيها الذين آمنوا لا ترفعوا أصواتكم فوق صوت النبي ولا تجهروا له بالقول كجهر بعضكم لبعض أن تحبط أعمالكم وأنتم لا تشعرون
\end{taggedblock}
\begin{taggedblock}{variations_in_reading}
%This section is optional, for translating different wordings. For each different wording, paste the translation again, with the changes from Hafṣ highlighted in bold.
\end{taggedblock}
\begin{taggedblock}{translator_comments}
%Put any comments that you have as a translator, including issues of concern, or major decisions that you made when translating.
\end{taggedblock}
\begin{taggedblock}{aayah_tags}
%Put tags here separated by commas, e.g.: tawheed,prophets,yusuf,dua
\end{taggedblock}
\begin{taggedblock}{tafseer_sadi_arabic}
\begin{Arabic}
{ يَا أَيُّهَا الَّذِينَ آمَنُوا لَا تَرْفَعُوا أَصْوَاتَكُمْ فَوْقَ صَوْتِ النَّبِيِّ وَلَا تَجْهَرُوا لَهُ بِالْقَوْلِ }
وهذا أدب مع رسول الله صلى الله عليه وسلم، في خطابه، أي: لا يرفع المخاطب له، صوته معه، فوق صوته، ولا يجهر له بالقول، بل يغض الصوت، ويخاطبه بأدب ولين، وتعظيم وتكريم، وإجلال وإعظام، ولا يكون الرسول كأحدهم، بل يميزوه في خطابهم، كما تميز عن غيره، في وجوب حقه على الأمة، ووجوب الإيمان به، والحب الذي لا يتم الإيمان إلا به، فإن في عدم القيام بذلك، محذورًا، وخشية أن يحبط عمل العبد وهو لا يشعر، كما أن الأدب معه، من أسباب
[حصول الثواب و]
قبول الأعمال.
\end{Arabic}
\end{taggedblock}
\begin{taggedblock}{tafseer_sadi_translation}
{ يَا أَيُّهَا الَّذِينَ آمَنُوا لَا تَرْفَعُوا أَصْوَاتَكُمْ فَوْقَ صَوْتِ النَّبِيِّ وَلَا تَجْهَرُوا لَهُ بِالْقَوْلِ }
وهذا أدب مع رسول الله صلى الله عليه وسلم، في خطابه، أي: لا يرفع المخاطب له، صوته معه، فوق صوته، ولا يجهر له بالقول، بل يغض الصوت، ويخاطبه بأدب ولين، وتعظيم وتكريم، وإجلال وإعظام، ولا يكون الرسول كأحدهم، بل يميزوه في خطابهم، كما تميز عن غيره، في وجوب حقه على الأمة، ووجوب الإيمان به، والحب الذي لا يتم الإيمان إلا به، فإن في عدم القيام بذلك، محذورًا، وخشية أن يحبط عمل العبد وهو لا يشعر، كما أن الأدب معه، من أسباب
[حصول الثواب و]
قبول الأعمال.
\end{taggedblock}
\begin{taggedblock}{license_attribution_aayah}
\input{license-attribution}
\end{taggedblock}
\begin{comment}
Please use the following for footnotes:- Sample\footnoteQ{Text of Qur'an footnote goes here.}.
Sample\footnoteT{Text of Tafseer footnote goes here.}.
\end{comment}