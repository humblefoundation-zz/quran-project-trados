%This file is used by a ruby script as a template for each aayah
\begin{comment}
The following strings are to be replaced by a script, in order to use this file as a template (all upper case):-
* sN = sūrah number, without leading zeros
* aYN = āyah number, without leading zeros
* aRABIC_TEXT_TASHKEEL = the text of the aayah, with tashkeel marks
* aRABIC_TEXT_WITHOUT_TASHKEEL = the text of the aayah, without tashkeel marks
* tAFSEER_SADI_ARABIC = the tafseer of the aayah from as-sa'di
\end{comment}
\begin{comment}
The following tags are declared here:-
quran_com_link
quran_com_arabic_image
arabic_text_tashkeel
our_translation
variations_in_reading
translator_comments
aayah_tags
tafseer_sadi_arabic
tafseer_sadi_translation
\end{comment}
\begin{taggedblock}{quran_com_link}
\href{http://quran.com/49/10}{Link to 49:10 on quran.com}
\end{taggedblock}
\begin{taggedblock}{quran_com_arabic_image}
\includegraphics{49_10}
\end{taggedblock}
\begin{taggedblock}{arabic_text_tashkeel}
\begin{Arabic}
إِنَّمَا الْمُؤْمِنُونَ إِخْوَةٌ فَأَصْلِحُوا بَيْنَ أَخَوَيْكُمْ وَاتَّقُوا اللَّهَ لَعَلَّكُمْ تُرْحَمُونَ
\end{Arabic}
\end{taggedblock}
\begin{taggedblock}{our_translation}
إنما المؤمنون إخوة فأصلحوا بين أخويكم واتقوا الله لعلكم ترحمون
\end{taggedblock}
\begin{taggedblock}{variations_in_reading}
%This section is optional, for translating different wordings. For each different wording, paste the translation again, with the changes from Hafṣ highlighted in bold.
\end{taggedblock}
\begin{taggedblock}{translator_comments}
%Put any comments that you have as a translator, including issues of concern, or major decisions that you made when translating.
\end{taggedblock}
\begin{taggedblock}{aayah_tags}
%Put tags here separated by commas, e.g.: tawheed,prophets,yusuf,dua
\end{taggedblock}
\begin{taggedblock}{tafseer_sadi_arabic}
\begin{Arabic}
{ إِنَّمَا الْمُؤْمِنُونَ إِخْوَةٌ }
هذا عقد، عقده الله بين المؤمنين، أنه إذا وجد من أي شخص كان، في مشرق الأرض ومغربها، الإيمان بالله، وملائكته، وكتبه، ورسله، واليوم الآخر، فإنه أخ للمؤمنين، أخوة توجب أن يحب له المؤمنون، ما يحبون لأنفسهم، ويكرهون له، ما يكرهون لأنفسهم، ولهذا قال النبي صلى الله عليه وسلم آمرًا بحقوق الأخوة الإيمانية:
"لا تحاسدوا، ولا تناجشوا، ولا تباغضوا، ولا يبع أحدكم على بيع بعض، وكونوا عباد الله إخوانًا المؤمن أخو المؤمن، لا يظلمه، ولا يخذله، ولا يحقره"

وقال صلى الله عليه وسلم
"المؤمن للمؤمن، كالبنيان يشد بعضه بعضًا"
وشبك صلى الله عليه وسلم بين أصابعه.

ولقد أمر الله ورسوله، بالقيام بحقوق المؤمنين، بعضهم لبعض، وبما به يحصل التآلف والتوادد، والتواصل بينهم، كل هذا، تأييد لحقوق بعضهم على بعض، فمن ذلك، إذا وقع الاقتتال بينهم، الموجب لتفرق القلوب وتباغضها
[وتدابرها]
، فليصلح المؤمنون بين إخوانهم، وليسعوا فيما به يزول شنآنهم.

ثم أمر بالتقوى عمومًا، ورتب على القيام بحقوق المؤمنين وبتقوى الله، الرحمة [ فقال:
{ لَعَلَّكُمْ تُرْحَمُونَ }
وإذا حصلت الرحمة، حصل خير الدنيا والآخرة، ودل ذلك، على أن عدم القيام بحقوق المؤمنين، من أعظم حواجب الرحمة.

وفي هاتين الآيتين من الفوائد، غير ما تقدم: أن الاقتتال بين المؤمنين مناف للأخوة الإيمانية، ولهذا، كان من أكبر الكبائر، وأن الإيمان، والأخوة الإيمانية، لا تزول مع وجود القتال كغيره من الذنوب الكبار، التي دون الشرك، وعلى ذلك مذهب أهل السنة والجماعة، وعلى وجوب الإصلاح، بين المؤمنين بالعدل، وعلى وجوب قتال البغاة، حتى يرجعوا إلى أمر الله، وعلى أنهم لو رجعوا، لغير أمر الله، بأن رجعوا على وجه لا يجوز الإقرار عليه والتزامه، أنه لا يجوز ذلك، وأن أموالهم معصومة، لأن الله أباح دماءهم وقت استمرارهم على بغيهم خاصة، دون أموالهم.
\end{Arabic}
\end{taggedblock}
\begin{taggedblock}{tafseer_sadi_translation}
{ إِنَّمَا الْمُؤْمِنُونَ إِخْوَةٌ }
هذا عقد، عقده الله بين المؤمنين، أنه إذا وجد من أي شخص كان، في مشرق الأرض ومغربها، الإيمان بالله، وملائكته، وكتبه، ورسله، واليوم الآخر، فإنه أخ للمؤمنين، أخوة توجب أن يحب له المؤمنون، ما يحبون لأنفسهم، ويكرهون له، ما يكرهون لأنفسهم، ولهذا قال النبي صلى الله عليه وسلم آمرًا بحقوق الأخوة الإيمانية:
"لا تحاسدوا، ولا تناجشوا، ولا تباغضوا، ولا يبع أحدكم على بيع بعض، وكونوا عباد الله إخوانًا المؤمن أخو المؤمن، لا يظلمه، ولا يخذله، ولا يحقره"

وقال صلى الله عليه وسلم
"المؤمن للمؤمن، كالبنيان يشد بعضه بعضًا"
وشبك صلى الله عليه وسلم بين أصابعه.

ولقد أمر الله ورسوله، بالقيام بحقوق المؤمنين، بعضهم لبعض، وبما به يحصل التآلف والتوادد، والتواصل بينهم، كل هذا، تأييد لحقوق بعضهم على بعض، فمن ذلك، إذا وقع الاقتتال بينهم، الموجب لتفرق القلوب وتباغضها
[وتدابرها]
، فليصلح المؤمنون بين إخوانهم، وليسعوا فيما به يزول شنآنهم.

ثم أمر بالتقوى عمومًا، ورتب على القيام بحقوق المؤمنين وبتقوى الله، الرحمة [ فقال:
{ لَعَلَّكُمْ تُرْحَمُونَ }
وإذا حصلت الرحمة، حصل خير الدنيا والآخرة، ودل ذلك، على أن عدم القيام بحقوق المؤمنين، من أعظم حواجب الرحمة.

وفي هاتين الآيتين من الفوائد، غير ما تقدم: أن الاقتتال بين المؤمنين مناف للأخوة الإيمانية، ولهذا، كان من أكبر الكبائر، وأن الإيمان، والأخوة الإيمانية، لا تزول مع وجود القتال كغيره من الذنوب الكبار، التي دون الشرك، وعلى ذلك مذهب أهل السنة والجماعة، وعلى وجوب الإصلاح، بين المؤمنين بالعدل، وعلى وجوب قتال البغاة، حتى يرجعوا إلى أمر الله، وعلى أنهم لو رجعوا، لغير أمر الله، بأن رجعوا على وجه لا يجوز الإقرار عليه والتزامه، أنه لا يجوز ذلك، وأن أموالهم معصومة، لأن الله أباح دماءهم وقت استمرارهم على بغيهم خاصة، دون أموالهم.
\end{taggedblock}
\input{license-attribution}
\begin{comment}
Please use the following for footnotes:- Sample\footnoteQ{Text of Qur'an footnote goes here.}.
Sample\footnoteT{Text of Tafseer footnote goes here.}.
\end{comment}