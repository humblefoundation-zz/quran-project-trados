%This file is used by a ruby script as a template for each aayah
\begin{comment}
The following strings are to be replaced by a script, in order to use this file as a template (all upper case):-
* sN = sūrah number, without leading zeros
* aYN = āyah number, without leading zeros
* aRABIC_TEXT_TASHKEEL = the text of the aayah, with tashkeel marks
* aRABIC_TEXT_WITHOUT_TASHKEEL = the text of the aayah, without tashkeel marks
* tAFSEER_SADI_ARABIC = the tafseer of the aayah from as-sa'di
\end{comment}
\begin{comment}
The following tags are declared here:-
quran_com_link
quran_com_arabic_image
arabic_text_tashkeel
our_translation
variations_in_reading
translator_comments
aayah_tags
tafseer_sadi_arabic
tafseer_sadi_translation
\end{comment}
\begin{taggedblock}{quran_com_link}
\href{http://quran.com/49/7}{Link to 49:7 on quran.com}
\end{taggedblock}
\begin{taggedblock}{quran_com_arabic_image}
\includegraphics{49_7}
\end{taggedblock}
\begin{taggedblock}{arabic_text_tashkeel}
\begin{Arabic}
وَاعْلَمُوا أَنَّ فِيكُمْ رَسُولَ اللَّهِ لَوْ يُطِيعُكُمْ فِي كَثِيرٍ مِنَ الْأَمْرِ لَعَنِتُّمْ وَلَٰكِنَّ اللَّهَ حَبَّبَ إِلَيْكُمُ الْإِيمَانَ وَزَيَّنَهُ فِي قُلُوبِكُمْ وَكَرَّهَ إِلَيْكُمُ الْكُفْرَ وَالْفُسُوقَ وَالْعِصْيَانَ أُولَٰئِكَ هُمُ الرَّاشِدُونَ
\end{Arabic}
\end{taggedblock}
\begin{taggedblock}{our_translation}
واعلموا أن فيكم رسول الله لو يطيعكم في كثير من الأمر لعنتم ولكن الله حبب إليكم الإيمان وزينه في قلوبكم وكره إليكم الكفر والفسوق والعصيان أولئك هم الراشدون
\end{taggedblock}
\begin{taggedblock}{variations_in_reading}
%This section is optional, for translating different wordings. For each different wording, paste the translation again, with the changes from Hafṣ highlighted in bold.
\end{taggedblock}
\begin{taggedblock}{translator_comments}
%Put any comments that you have as a translator, including issues of concern, or major decisions that you made when translating.
\end{taggedblock}
\begin{taggedblock}{aayah_tags}
%Put tags here separated by commas, e.g.: tawheed,prophets,yusuf,dua
\end{taggedblock}
\begin{taggedblock}{tafseer_sadi_arabic}
\begin{Arabic}
أي: ليكن لديكم معلومًا أن رسول الله صلى الله عليه وسلم، بين أظهركم، وهو الرسول الكريم، البار، الراشد، الذي يريد بكم الخير وينصح لكم، وتريدون لأنفسكم من الشر والمضرة، ما لا يوافقكم الرسول عليه، ولو يطيعكم في كثير من الأمر لشق عليكم وأعنتكم، ولكن الرسول يرشدكم، والله تعالى يحبب إليكم الإيمان، ويزينه في قلوبكم، بما أودع الله في قلوبكم من محبة الحق وإيثاره، وبما ينصب على الحق من الشواهد، والأدلة الدالة على صحته، وقبول القلوب والفطر له، وبما يفعله تعالى بكم، من توفيقه للإنابة إليه، ويكره إليكم الكفر والفسوق، أي: الذنوب الكبار، والعصيان: هي ما دون ذلك من الذنوب  بما أودع في قلوبكم من كراهة الشر، وعدم إرادة فعله، وبما نصبه من الأدلة والشواهد على فساده، وعدم قبول الفطر له، وبما يجعله الله من الكراهة في القلوب له

{ أُولَئِكَ }
أي: الذين زين الله الإيمان في قلوبهم، وحببه إليهم، وكره إليهم الكفر والفسوق والعصيان
{ هُمُ الرَّاشِدُونَ }
أي: الذين صلحت علومهم وأعمالهم، واستقاموا على الدين القويم، والصراط المستقيم.

وضدهم الغاوون، الذين حبب إليهم الكفر والفسوق والعصيان، وكره إليهم الإيمان، والذنب ذنبهم، فإنهم لما فسقوا طبع الله على قلوبهم، ولما
{ زَاغُوا أَزَاغَ اللَّهُ قُلُوبَهُمْ }
ولما لم يؤمنوا بالحق لما جاءهم أول مرة، قلب الله أفئدتهم.
\end{Arabic}
\end{taggedblock}
\begin{taggedblock}{tafseer_sadi_translation}
أي: ليكن لديكم معلومًا أن رسول الله صلى الله عليه وسلم، بين أظهركم، وهو الرسول الكريم، البار، الراشد، الذي يريد بكم الخير وينصح لكم، وتريدون لأنفسكم من الشر والمضرة، ما لا يوافقكم الرسول عليه، ولو يطيعكم في كثير من الأمر لشق عليكم وأعنتكم، ولكن الرسول يرشدكم، والله تعالى يحبب إليكم الإيمان، ويزينه في قلوبكم، بما أودع الله في قلوبكم من محبة الحق وإيثاره، وبما ينصب على الحق من الشواهد، والأدلة الدالة على صحته، وقبول القلوب والفطر له، وبما يفعله تعالى بكم، من توفيقه للإنابة إليه، ويكره إليكم الكفر والفسوق، أي: الذنوب الكبار، والعصيان: هي ما دون ذلك من الذنوب  بما أودع في قلوبكم من كراهة الشر، وعدم إرادة فعله، وبما نصبه من الأدلة والشواهد على فساده، وعدم قبول الفطر له، وبما يجعله الله من الكراهة في القلوب له

{ أُولَئِكَ }
أي: الذين زين الله الإيمان في قلوبهم، وحببه إليهم، وكره إليهم الكفر والفسوق والعصيان
{ هُمُ الرَّاشِدُونَ }
أي: الذين صلحت علومهم وأعمالهم، واستقاموا على الدين القويم، والصراط المستقيم.

وضدهم الغاوون، الذين حبب إليهم الكفر والفسوق والعصيان، وكره إليهم الإيمان، والذنب ذنبهم، فإنهم لما فسقوا طبع الله على قلوبهم، ولما
{ زَاغُوا أَزَاغَ اللَّهُ قُلُوبَهُمْ }
ولما لم يؤمنوا بالحق لما جاءهم أول مرة، قلب الله أفئدتهم.
\end{taggedblock}
\input{license-attribution}
\begin{comment}
Please use the following for footnotes:- Sample\footnoteQ{Text of Qur'an footnote goes here.}.
Sample\footnoteT{Text of Tafseer footnote goes here.}.
\end{comment}