%This file is used by a ruby script as a template for each aayah
\begin{comment}
The following strings are to be replaced by a script, in order to use this file as a template (all upper case):-
* sN = sūrah number, without leading zeros
* aYN = āyah number, without leading zeros
* aRABIC_TEXT_TASHKEEL = the text of the aayah, with tashkeel marks
* aRABIC_TEXT_WITHOUT_TASHKEEL = the text of the aayah, without tashkeel marks
* tAFSEER_SADI_ARABIC = the tafseer of the aayah from as-sa'di
\end{comment}
\begin{comment}
The following tags are declared here:-
quran_com_link
quran_com_arabic_image
arabic_text_tashkeel
our_translation
variations_in_reading
translator_comments
aayah_tags
tafseer_sadi_arabic
tafseer_sadi_translation
license_attribution_aayah
\end{comment}
\begin{taggedblock}{quran_com_link}
\href{http://quran.com/49/15}{Link to 49:15 on quran.com}
\end{taggedblock}
\begin{taggedblock}{quran_com_arabic_image}
\includegraphics{49_15}
\end{taggedblock}
\begin{taggedblock}{arabic_text_tashkeel}
\begin{Arabic}
إِنَّمَا الْمُؤْمِنُونَ الَّذِينَ آمَنُوا بِاللَّهِ وَرَسُولِهِ ثُمَّ لَمْ يَرْتَابُوا وَجَاهَدُوا بِأَمْوَالِهِمْ وَأَنْفُسِهِمْ فِي سَبِيلِ اللَّهِ أُولَٰئِكَ هُمُ الصَّادِقُونَ
\end{Arabic}
\end{taggedblock}
\begin{taggedblock}{our_translation}
إنما المؤمنون الذين آمنوا بالله ورسوله ثم لم يرتابوا وجاهدوا بأموالهم وأنفسهم في سبيل الله أولئك هم الصادقون
\end{taggedblock}
\begin{taggedblock}{variations_in_reading}
%This section is optional, for translating different wordings. For each different wording, paste the translation again, with the changes from Hafṣ highlighted in bold.
\end{taggedblock}
\begin{taggedblock}{translator_comments}
%Put any comments that you have as a translator, including issues of concern, or major decisions that you made when translating.
\end{taggedblock}
\begin{taggedblock}{aayah_tags}
%Put tags here separated by commas, e.g.: tawheed,prophets,yusuf,dua
\end{taggedblock}
\begin{taggedblock}{tafseer_sadi_arabic}
\begin{Arabic}
{ إِنَّمَا الْمُؤْمِنُونَ }
أي: على الحقيقة
{ الَّذِينَ آمَنُوا بِاللَّهِ وَرَسُولِهِ ثُمَّ لَمْ يَرتَابُوا وَجَاهَدُوا بِأمْوَالِهِمْ وَأنْفُسِهُم في سبيل الله }
أي: من جمعوا بين الإيمان والجهاد في سبيله، فإن من جاهد الكفار، دل ذلك، على الإيمان التام في القلب، لأن من جاهد غيره على الإسلام، والقيام بشرائعه، فجهاده لنفسه على ذلك، من باب أولى وأحرى؛ ولأن من لم يقو على الجهاد، فإن ذلك، دليل على ضعف إيمانه، وشرط تعالى في الإيمان عدم الريب، وهو الشك، لأن الإيمان النافع هو الجزم اليقيني، بما أمر الله بالإيمان به، الذي لا يعتريه شك، بوجه من الوجوه.

وقوله:
{ أُولَئِكَ هُمُ الصَّادِقُونَ }
أي: الذين صدقوا إيمانهم بأعمالهم الجميلة، فإن الصدق، دعوى كبيرة في كل شيء يدعى يحتاج صاحبه إلى حجة وبرهان، وأعظم ذلك، دعوى الإيمان، الذي هو مدار السعادة، والفوز الأبدي، والفلاح السرمدي، فمن ادعاه، وقام بواجباته، ولوازمه، فهو الصادق المؤمن حقًا، ومن لم يكن كذلك، علم أنه ليس بصادق في دعواه، وليس لدعواه فائدة، فإن الإيمان في القلب لا يطلع عليه إلا الله تعالى.

فإثباته ونفيه، من باب تعليم الله بما في القلب، وهذا سوء أدب، وظن بالله.
\end{Arabic}
\end{taggedblock}
\begin{taggedblock}{tafseer_sadi_translation}
{ إِنَّمَا الْمُؤْمِنُونَ }
أي: على الحقيقة
{ الَّذِينَ آمَنُوا بِاللَّهِ وَرَسُولِهِ ثُمَّ لَمْ يَرتَابُوا وَجَاهَدُوا بِأمْوَالِهِمْ وَأنْفُسِهُم في سبيل الله }
أي: من جمعوا بين الإيمان والجهاد في سبيله، فإن من جاهد الكفار، دل ذلك، على الإيمان التام في القلب، لأن من جاهد غيره على الإسلام، والقيام بشرائعه، فجهاده لنفسه على ذلك، من باب أولى وأحرى؛ ولأن من لم يقو على الجهاد، فإن ذلك، دليل على ضعف إيمانه، وشرط تعالى في الإيمان عدم الريب، وهو الشك، لأن الإيمان النافع هو الجزم اليقيني، بما أمر الله بالإيمان به، الذي لا يعتريه شك، بوجه من الوجوه.

وقوله:
{ أُولَئِكَ هُمُ الصَّادِقُونَ }
أي: الذين صدقوا إيمانهم بأعمالهم الجميلة، فإن الصدق، دعوى كبيرة في كل شيء يدعى يحتاج صاحبه إلى حجة وبرهان، وأعظم ذلك، دعوى الإيمان، الذي هو مدار السعادة، والفوز الأبدي، والفلاح السرمدي، فمن ادعاه، وقام بواجباته، ولوازمه، فهو الصادق المؤمن حقًا، ومن لم يكن كذلك، علم أنه ليس بصادق في دعواه، وليس لدعواه فائدة، فإن الإيمان في القلب لا يطلع عليه إلا الله تعالى.

فإثباته ونفيه، من باب تعليم الله بما في القلب، وهذا سوء أدب، وظن بالله.
\end{taggedblock}
\begin{taggedblock}{license_attribution_aayah}
\input{license-attribution}
\end{taggedblock}
\begin{comment}
Please use the following for footnotes:- Sample\footnoteQ{Text of Qur'an footnote goes here.}.
Sample\footnoteT{Text of Tafseer footnote goes here.}.
\end{comment}