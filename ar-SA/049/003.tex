%This file is used by a ruby script as a template for each aayah
\begin{comment}
The following strings are to be replaced by a script, in order to use this file as a template (all upper case):-
* sN = sūrah number, without leading zeros
* aYN = āyah number, without leading zeros
* aRABIC_TEXT_TASHKEEL = the text of the aayah, with tashkeel marks
* aRABIC_TEXT_WITHOUT_TASHKEEL = the text of the aayah, without tashkeel marks
* tAFSEER_SADI_ARABIC = the tafseer of the aayah from as-sa'di
\end{comment}
\begin{comment}
The following tags are declared here:-
quran_com_link
quran_com_arabic_image
arabic_text_tashkeel
our_translation
variations_in_reading
translator_comments
aayah_tags
tafseer_sadi_arabic
tafseer_sadi_translation
license_attribution_aayah
\end{comment}
\begin{taggedblock}{quran_com_link}
\href{http://quran.com/49/3}{Link to 49:3 on quran.com}
\end{taggedblock}
\begin{taggedblock}{quran_com_arabic_image}
\includegraphics{49_3}
\end{taggedblock}
\begin{taggedblock}{arabic_text_tashkeel}
\begin{Arabic}
إِنَّ الَّذِينَ يَغُضُّونَ أَصْوَاتَهُمْ عِنْدَ رَسُولِ اللَّهِ أُولَٰئِكَ الَّذِينَ امْتَحَنَ اللَّهُ قُلُوبَهُمْ لِلتَّقْوَىٰ لَهُمْ مَغْفِرَةٌ وَأَجْرٌ عَظِيمٌ
\end{Arabic}
\end{taggedblock}
\begin{taggedblock}{our_translation}
إن الذين يغضون أصواتهم عند رسول الله أولئك الذين امتحن الله قلوبهم للتقوى لهم مغفرة وأجر عظيم
\end{taggedblock}
\begin{taggedblock}{variations_in_reading}
%This section is optional, for translating different wordings. For each different wording, paste the translation again, with the changes from Hafṣ highlighted in bold.
\end{taggedblock}
\begin{taggedblock}{translator_comments}
%Put any comments that you have as a translator, including issues of concern, or major decisions that you made when translating.
\end{taggedblock}
\begin{taggedblock}{aayah_tags}
%Put tags here separated by commas, e.g.: tawheed,prophets,yusuf,dua
\end{taggedblock}
\begin{taggedblock}{tafseer_sadi_arabic}
\begin{Arabic}
ثم مدح من غض صوته عند رسول الله صلى الله عليه وسلم، بأن الله امتحن قلوبهم للتقوى، أي: ابتلاها واختبرها، فظهرت نتيجة ذلك، بأن صلحت قلوبهم للتقوى، ثم وعدهم المغفرة لذنوبهم، المتضمنة لزوال الشر والمكروه، والأجر العظيم، الذي لا يعلم وصفه إلا الله تعالى، وفي الأجر العظيم وجود المحبوب  وفي هذا، دليل على أن الله يمتحن القلوب، بالأمر والنهي والمحن، فمن لازم أمر الله، واتبع رضاه، وسارع إلى ذلك، وقدمه على هواه، تمحض وتمحص للتقوى، وصار قلبه صالحًا لها ومن لم يكن كذلك، علم أنه لا يصلح للتقوى.
\end{Arabic}
\end{taggedblock}
\begin{taggedblock}{tafseer_sadi_translation}
ثم مدح من غض صوته عند رسول الله صلى الله عليه وسلم، بأن الله امتحن قلوبهم للتقوى، أي: ابتلاها واختبرها، فظهرت نتيجة ذلك، بأن صلحت قلوبهم للتقوى، ثم وعدهم المغفرة لذنوبهم، المتضمنة لزوال الشر والمكروه، والأجر العظيم، الذي لا يعلم وصفه إلا الله تعالى، وفي الأجر العظيم وجود المحبوب  وفي هذا، دليل على أن الله يمتحن القلوب، بالأمر والنهي والمحن، فمن لازم أمر الله، واتبع رضاه، وسارع إلى ذلك، وقدمه على هواه، تمحض وتمحص للتقوى، وصار قلبه صالحًا لها ومن لم يكن كذلك، علم أنه لا يصلح للتقوى.
\end{taggedblock}
\begin{taggedblock}{license_attribution_aayah}
\input{license-attribution}
\end{taggedblock}
\begin{comment}
Please use the following for footnotes:- Sample\footnoteQ{Text of Qur'an footnote goes here.}.
Sample\footnoteT{Text of Tafseer footnote goes here.}.
\end{comment}