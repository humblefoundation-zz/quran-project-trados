%This file is used by a ruby script as a template for each aayah
\begin{comment}
The following strings are to be replaced by a script, in order to use this file as a template (all upper case):-
* sN = sūrah number, without leading zeros
* aYN = āyah number, without leading zeros
* aRABIC_TEXT_TASHKEEL = the text of the aayah, with tashkeel marks
* aRABIC_TEXT_WITHOUT_TASHKEEL = the text of the aayah, without tashkeel marks
* tAFSEER_SADI_ARABIC = the tafseer of the aayah from as-sa'di
\end{comment}
\begin{comment}
The following tags are declared here:-
quran_com_link
quran_com_arabic_image
arabic_text_tashkeel
our_translation
variations_in_reading
translator_comments
aayah_tags
tafseer_sadi_arabic
tafseer_sadi_translation
license_attribution_aayah
\end{comment}
\begin{taggedblock}{quran_com_link}
\href{http://quran.com/49/6}{Link to 49:6 on quran.com}
\end{taggedblock}
\begin{taggedblock}{quran_com_arabic_image}
\includegraphics{49_6}
\end{taggedblock}
\begin{taggedblock}{arabic_text_tashkeel}
\begin{Arabic}
يَا أَيُّهَا الَّذِينَ آمَنُوا إِنْ جَاءَكُمْ فَاسِقٌ بِنَبَإٍ فَتَبَيَّنُوا أَنْ تُصِيبُوا قَوْمًا بِجَهَالَةٍ فَتُصْبِحُوا عَلَىٰ مَا فَعَلْتُمْ نَادِمِينَ
\end{Arabic}
\end{taggedblock}
\begin{taggedblock}{our_translation}
يا أيها الذين آمنوا إن جاءكم فاسق بنبإ فتبينوا أن تصيبوا قوما بجهالة فتصبحوا على ما فعلتم نادمين
\end{taggedblock}
\begin{taggedblock}{variations_in_reading}
%This section is optional, for translating different wordings. For each different wording, paste the translation again, with the changes from Hafṣ highlighted in bold.
\end{taggedblock}
\begin{taggedblock}{translator_comments}
%Put any comments that you have as a translator, including issues of concern, or major decisions that you made when translating.
\end{taggedblock}
\begin{taggedblock}{aayah_tags}
%Put tags here separated by commas, e.g.: tawheed,prophets,yusuf,dua
\end{taggedblock}
\begin{taggedblock}{tafseer_sadi_arabic}
\begin{Arabic}
وهذا أيضًا، من الآداب التي على أولي الألباب، التأدب بها واستعمالها، وهو أنه إذا أخبرهم فاسق بخبر أن يتثبتوا في خبره، ولا يأخذوه مجردًا، فإن في ذلك خطرًا كبيرًا، ووقوعًا في الإثم، فإن خبره إذا جعل بمنزلة خبر الصادق العدل، حكم بموجب ذلك ومقتضاه، فحصل من تلف النفوس والأموال، بغير حق، بسبب ذلك الخبر ما يكون سببًا للندامة، بل الواجب عند خبر الفاسق، التثبت والتبين، فإن دلت الدلائل والقرائن على صدقه، عمل به وصدق، وإن دلت على كذبه، كذب، ولم يعمل به، ففيه دليل، على أن خبر الصادق مقبول، وخبر الكاذب، مردود، وخبر الفاسق متوقف فيه كما ذكرنا، ولهذا كان السلف يقبلون روايات كثير
[من]
الخوارج، المعروفين بالصدق، ولو كانوا فساقًا.
\end{Arabic}
\end{taggedblock}
\begin{taggedblock}{tafseer_sadi_translation}
وهذا أيضًا، من الآداب التي على أولي الألباب، التأدب بها واستعمالها، وهو أنه إذا أخبرهم فاسق بخبر أن يتثبتوا في خبره، ولا يأخذوه مجردًا، فإن في ذلك خطرًا كبيرًا، ووقوعًا في الإثم، فإن خبره إذا جعل بمنزلة خبر الصادق العدل، حكم بموجب ذلك ومقتضاه، فحصل من تلف النفوس والأموال، بغير حق، بسبب ذلك الخبر ما يكون سببًا للندامة، بل الواجب عند خبر الفاسق، التثبت والتبين، فإن دلت الدلائل والقرائن على صدقه، عمل به وصدق، وإن دلت على كذبه، كذب، ولم يعمل به، ففيه دليل، على أن خبر الصادق مقبول، وخبر الكاذب، مردود، وخبر الفاسق متوقف فيه كما ذكرنا، ولهذا كان السلف يقبلون روايات كثير
[من]
الخوارج، المعروفين بالصدق، ولو كانوا فساقًا.
\end{taggedblock}
\begin{taggedblock}{license_attribution_aayah}
\input{license-attribution}
\end{taggedblock}
\begin{comment}
Please use the following for footnotes:- Sample\footnoteQ{Text of Qur'an footnote goes here.}.
Sample\footnoteT{Text of Tafseer footnote goes here.}.
\end{comment}