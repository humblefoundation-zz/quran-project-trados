%This file is used by a ruby script as a template for each aayah
\begin{comment}
The following strings are to be replaced by a script, in order to use this file as a template (all upper case):-
* sN = sūrah number, without leading zeros
* aYN = āyah number, without leading zeros
* aRABIC_TEXT_TASHKEEL = the text of the aayah, with tashkeel marks
* aRABIC_TEXT_WITHOUT_TASHKEEL = the text of the aayah, without tashkeel marks
* tAFSEER_SADI_ARABIC = the tafseer of the aayah from as-sa'di
\end{comment}
\begin{comment}
The following tags are declared here:-
quran_com_link
quran_com_arabic_image
arabic_text_tashkeel
our_translation
variations_in_reading
translator_comments
aayah_tags
tafseer_sadi_arabic
tafseer_sadi_translation
\end{comment}
\begin{taggedblock}{quran_com_link}
\href{http://quran.com/49/4}{Link to 49:4 on quran.com}
\end{taggedblock}
\begin{taggedblock}{quran_com_arabic_image}
\includegraphics{49_4}
\end{taggedblock}
\begin{taggedblock}{arabic_text_tashkeel}
\begin{Arabic}
إِنَّ الَّذِينَ يُنَادُونَكَ مِنْ وَرَاءِ الْحُجُرَاتِ أَكْثَرُهُمْ لَا يَعْقِلُونَ
\end{Arabic}
\end{taggedblock}
\begin{taggedblock}{our_translation}
إن الذين ينادونك من وراء الحجرات أكثرهم لا يعقلون
\end{taggedblock}
\begin{taggedblock}{variations_in_reading}
%This section is optional, for translating different wordings. For each different wording, paste the translation again, with the changes from Hafṣ highlighted in bold.
\end{taggedblock}
\begin{taggedblock}{translator_comments}
%Put any comments that you have as a translator, including issues of concern, or major decisions that you made when translating.
\end{taggedblock}
\begin{taggedblock}{aayah_tags}
%Put tags here separated by commas, e.g.: tawheed,prophets,yusuf,dua
\end{taggedblock}
\begin{taggedblock}{tafseer_sadi_arabic}
\begin{Arabic}
نزلت هذه الآيات الكريمة، في أناس من الأعراب، الذين وصفهم الله تعالى بالجفاء، وأنهم أجدر أن لا يعلموا حدود ما أنزل الله على رسوله، قدموا وافدين على رسول الله صلى الله عليه وسلم، فوجدوه في بيته وحجرات نسائه، فلم يصبروا ويتأدبوا حتى يخرج، بل نادوه: يا محمد يا محمد،
[أي: اخرج إلينا]
، فذمهم الله بعدم العقل، حيث لم يعقلوا عن الله الأدب مع رسوله واحترامه، كما أن من العقل وعلامته استعمال الأدب.
\end{Arabic}
\end{taggedblock}
\begin{taggedblock}{tafseer_sadi_translation}
نزلت هذه الآيات الكريمة، في أناس من الأعراب، الذين وصفهم الله تعالى بالجفاء، وأنهم أجدر أن لا يعلموا حدود ما أنزل الله على رسوله، قدموا وافدين على رسول الله صلى الله عليه وسلم، فوجدوه في بيته وحجرات نسائه، فلم يصبروا ويتأدبوا حتى يخرج، بل نادوه: يا محمد يا محمد،
[أي: اخرج إلينا]
، فذمهم الله بعدم العقل، حيث لم يعقلوا عن الله الأدب مع رسوله واحترامه، كما أن من العقل وعلامته استعمال الأدب.
\end{taggedblock}
\input{license-attribution}
\begin{comment}
Please use the following for footnotes:- Sample\footnoteQ{Text of Qur'an footnote goes here.}.
Sample\footnoteT{Text of Tafseer footnote goes here.}.
\end{comment}