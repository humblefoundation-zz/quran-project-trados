%This file is used by a ruby script as a template for each aayah
\begin{comment}
The following strings are to be replaced by a script, in order to use this file as a template (all upper case):-
* sN = sūrah number, without leading zeros
* aYN = āyah number, without leading zeros
* aRABIC_TEXT_TASHKEEL = the text of the aayah, with tashkeel marks
* aRABIC_TEXT_WITHOUT_TASHKEEL = the text of the aayah, without tashkeel marks
* tAFSEER_SADI_ARABIC = the tafseer of the aayah from as-sa'di
\end{comment}
\begin{comment}
The following tags are declared here:-
quran_com_link
quran_com_arabic_image
arabic_text_tashkeel
our_translation
variations_in_reading
translator_comments
aayah_tags
tafseer_sadi_arabic
tafseer_sadi_translation
license_attribution_aayah
\end{comment}
\begin{taggedblock}{quran_com_link}
\href{http://quran.com/49/18}{Link to 49:18 on quran.com}
\end{taggedblock}
\begin{taggedblock}{quran_com_arabic_image}
\includegraphics{49_18}
\end{taggedblock}
\begin{taggedblock}{arabic_text_tashkeel}
\begin{Arabic}
إِنَّ اللَّهَ يَعْلَمُ غَيْبَ السَّمَاوَاتِ وَالْأَرْضِ وَاللَّهُ بَصِيرٌ بِمَا تَعْمَلُونَ
\end{Arabic}
\end{taggedblock}
\begin{taggedblock}{our_translation}
إن الله يعلم غيب السماوات والأرض والله بصير بما تعملون
\end{taggedblock}
\begin{taggedblock}{variations_in_reading}
%This section is optional, for translating different wordings. For each different wording, paste the translation again, with the changes from Hafṣ highlighted in bold.
\end{taggedblock}
\begin{taggedblock}{translator_comments}
%Put any comments that you have as a translator, including issues of concern, or major decisions that you made when translating.
\end{taggedblock}
\begin{taggedblock}{aayah_tags}
%Put tags here separated by commas, e.g.: tawheed,prophets,yusuf,dua
\end{taggedblock}
\begin{taggedblock}{tafseer_sadi_arabic}
\begin{Arabic}
{ إِنَّ اللَّهَ يَعْلَمُ غَيْبَ السَّمَاوَاتِ وَالْأَرْضِ }
أي: الأمور الخفية فيهما، التي تخفى على الخلق، كالذي في لجج البحار، ومهامه القفار، وما جنه الليل أو واراه النهار، يعلم قطرات الأمطار، وحبات الرمال، ومكنونات الصدور، وخبايا الأمور.

{ وما تسقط من ورقة إلا يعلمها ولا حبة في ظلمات الأرض ولا رطب ولا يابس إلا في كتاب مبين }

{ وَاللَّهُ بَصِيرٌ بِمَا تَعْمَلُونَ }
يحصي عليكم أعمالكم، ويوفيكم إياها، ويجازيكم عليها بما تقتضيه رحمته الواسعة، وحكمته البالغة.

تم تفسير سورة الحجرات

بعون الله ومنه وجوده وكرمه،

فلك اللهم من الحمد أكمله وأتمه،

ومن الجود أفضله وأعمه
\end{Arabic}
\end{taggedblock}
\begin{taggedblock}{tafseer_sadi_translation}
{ إِنَّ اللَّهَ يَعْلَمُ غَيْبَ السَّمَاوَاتِ وَالْأَرْضِ }
أي: الأمور الخفية فيهما، التي تخفى على الخلق، كالذي في لجج البحار، ومهامه القفار، وما جنه الليل أو واراه النهار، يعلم قطرات الأمطار، وحبات الرمال، ومكنونات الصدور، وخبايا الأمور.

{ وما تسقط من ورقة إلا يعلمها ولا حبة في ظلمات الأرض ولا رطب ولا يابس إلا في كتاب مبين }

{ وَاللَّهُ بَصِيرٌ بِمَا تَعْمَلُونَ }
يحصي عليكم أعمالكم، ويوفيكم إياها، ويجازيكم عليها بما تقتضيه رحمته الواسعة، وحكمته البالغة.

تم تفسير سورة الحجرات

بعون الله ومنه وجوده وكرمه،

فلك اللهم من الحمد أكمله وأتمه،

ومن الجود أفضله وأعمه
\end{taggedblock}
\begin{taggedblock}{license_attribution_aayah}
\input{license-attribution}
\end{taggedblock}
\begin{comment}
Please use the following for footnotes:- Sample\footnoteQ{Text of Qur'an footnote goes here.}.
Sample\footnoteT{Text of Tafseer footnote goes here.}.
\end{comment}