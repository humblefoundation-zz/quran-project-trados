%This file is used by a ruby script as a template for each aayah
\begin{comment}
The following strings are to be replaced by a script, in order to use this file as a template (all upper case):-
* sN = sūrah number, without leading zeros
* aYN = āyah number, without leading zeros
* aRABIC_TEXT_TASHKEEL = the text of the aayah, with tashkeel marks
* aRABIC_TEXT_WITHOUT_TASHKEEL = the text of the aayah, without tashkeel marks
* tAFSEER_SADI_ARABIC = the tafseer of the aayah from as-sa'di
\end{comment}
\begin{comment}
The following tags are declared here:-
quran_com_link
quran_com_arabic_image
arabic_text_tashkeel
our_translation
variations_in_reading
translator_comments
aayah_tags
tafseer_sadi_arabic
tafseer_sadi_translation
\end{comment}
\begin{taggedblock}{quran_com_link}
\href{http://quran.com/49/12}{Link to 49:12 on quran.com}
\end{taggedblock}
\begin{taggedblock}{quran_com_arabic_image}
\includegraphics{49_12}
\end{taggedblock}
\begin{taggedblock}{arabic_text_tashkeel}
\begin{Arabic}
يَا أَيُّهَا الَّذِينَ آمَنُوا اجْتَنِبُوا كَثِيرًا مِنَ الظَّنِّ إِنَّ بَعْضَ الظَّنِّ إِثْمٌ وَلَا تَجَسَّسُوا وَلَا يَغْتَبْ بَعْضُكُمْ بَعْضًا أَيُحِبُّ أَحَدُكُمْ أَنْ يَأْكُلَ لَحْمَ أَخِيهِ مَيْتًا فَكَرِهْتُمُوهُ وَاتَّقُوا اللَّهَ إِنَّ اللَّهَ تَوَّابٌ رَحِيمٌ
\end{Arabic}
\end{taggedblock}
\begin{taggedblock}{our_translation}
يا أيها الذين آمنوا اجتنبوا كثيرا من الظن إن بعض الظن إثم ولا تجسسوا ولا يغتب بعضكم بعضا أيحب أحدكم أن يأكل لحم أخيه ميتا فكرهتموه واتقوا الله إن الله تواب رحيم
\end{taggedblock}
\begin{taggedblock}{variations_in_reading}
%This section is optional, for translating different wordings. For each different wording, paste the translation again, with the changes from Hafṣ highlighted in bold.
\end{taggedblock}
\begin{taggedblock}{translator_comments}
%Put any comments that you have as a translator, including issues of concern, or major decisions that you made when translating.
\end{taggedblock}
\begin{taggedblock}{aayah_tags}
%Put tags here separated by commas, e.g.: tawheed,prophets,yusuf,dua
\end{taggedblock}
\begin{taggedblock}{tafseer_sadi_arabic}
\begin{Arabic}
نهى الله تعالى عن كثير من الظن السوء  بالمؤمنين، فـ
{ إِنَّ بَعْضَ الظَّنِّ إِثْمٌ }
وذلك، كالظن الخالي من الحقيقة والقرينة، وكظن السوء، الذي يقترن به كثير من الأقوال، والأفعال المحرمة، فإن بقاء ظن السوء بالقلب، لا يقتصر صاحبه على مجرد ذلك، بل لا يزال به، حتى يقول ما لا ينبغي، ويفعل ما لا ينبغي، وفي ذلك أيضًا، إساءة الظن بالمسلم، وبغضه، وعداوته المأمور بخلاف ذلك منه.

{ وَلَا تَجَسَّسُوا }
أي: لا تفتشوا عن عورات المسلمين، ولا تتبعوها، واتركوا  المسلم على حاله، واستعملوا التغافل عن أحواله  التي إذا فتشت، ظهر منها ما لا ينبغي.

{ وَلَا يَغْتَبْ بَعْضُكُمْ بَعْضًا }
والغيبة، كما قال النبي صلى الله عليه وسلم:
{ ذكرك أخاك بما يكره ولو كان فيه }

ثم ذكر مثلاً منفرًا عن الغيبة، فقال:
{ أَيُحِبُّ أَحَدُكُمْ أَنْ يَأْكُلَ لَحْمَ أَخِيهِ مَيْتًا فَكَرِهْتُمُوهُ }
شبه أكل لحمه ميتًا، المكروه للنفوس
[غاية الكراهة]
، باغتيابه، فكما أنكم تكرهون أكل لحمه، وخصوصًا إذا كان ميتًا، فاقد الروح، فكذلك،
[فلتكرهوا]
غيبته، وأكل لحمه حيًا.

{ وَاتَّقُوا اللَّهَ إِنَّ اللَّهَ تَوَّابٌ رَحِيمٌ }
والتواب، الذي يأذن بتوبة عبده، فيوفقه لها، ثم يتوب عليه، بقبول توبته، رحيم بعباده، حيث دعاهم إلى ما ينفعهم، وقبل منهم التوبة، وفي هذه الآية، دليل على التحذير الشديد من الغيبة، وأن الغيبة من الكبائر، لأن الله شبهها بأكل لحم الميت، وذلك من الكبائر.
\end{Arabic}
\end{taggedblock}
\begin{taggedblock}{tafseer_sadi_translation}
نهى الله تعالى عن كثير من الظن السوء  بالمؤمنين، فـ
{ إِنَّ بَعْضَ الظَّنِّ إِثْمٌ }
وذلك، كالظن الخالي من الحقيقة والقرينة، وكظن السوء، الذي يقترن به كثير من الأقوال، والأفعال المحرمة، فإن بقاء ظن السوء بالقلب، لا يقتصر صاحبه على مجرد ذلك، بل لا يزال به، حتى يقول ما لا ينبغي، ويفعل ما لا ينبغي، وفي ذلك أيضًا، إساءة الظن بالمسلم، وبغضه، وعداوته المأمور بخلاف ذلك منه.

{ وَلَا تَجَسَّسُوا }
أي: لا تفتشوا عن عورات المسلمين، ولا تتبعوها، واتركوا  المسلم على حاله، واستعملوا التغافل عن أحواله  التي إذا فتشت، ظهر منها ما لا ينبغي.

{ وَلَا يَغْتَبْ بَعْضُكُمْ بَعْضًا }
والغيبة، كما قال النبي صلى الله عليه وسلم:
{ ذكرك أخاك بما يكره ولو كان فيه }

ثم ذكر مثلاً منفرًا عن الغيبة، فقال:
{ أَيُحِبُّ أَحَدُكُمْ أَنْ يَأْكُلَ لَحْمَ أَخِيهِ مَيْتًا فَكَرِهْتُمُوهُ }
شبه أكل لحمه ميتًا، المكروه للنفوس
[غاية الكراهة]
، باغتيابه، فكما أنكم تكرهون أكل لحمه، وخصوصًا إذا كان ميتًا، فاقد الروح، فكذلك،
[فلتكرهوا]
غيبته، وأكل لحمه حيًا.

{ وَاتَّقُوا اللَّهَ إِنَّ اللَّهَ تَوَّابٌ رَحِيمٌ }
والتواب، الذي يأذن بتوبة عبده، فيوفقه لها، ثم يتوب عليه، بقبول توبته، رحيم بعباده، حيث دعاهم إلى ما ينفعهم، وقبل منهم التوبة، وفي هذه الآية، دليل على التحذير الشديد من الغيبة، وأن الغيبة من الكبائر، لأن الله شبهها بأكل لحم الميت، وذلك من الكبائر.
\end{taggedblock}
\input{license-attribution}
\begin{comment}
Please use the following for footnotes:- Sample\footnoteQ{Text of Qur'an footnote goes here.}.
Sample\footnoteT{Text of Tafseer footnote goes here.}.
\end{comment}