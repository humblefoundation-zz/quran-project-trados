%This file is used by a ruby script as a template for each aayah
\begin{comment}
The following strings are to be replaced by a script, in order to use this file as a template (all upper case):-
* sN = sūrah number, without leading zeros
* aYN = āyah number, without leading zeros
* aRABIC_TEXT_TASHKEEL = the text of the aayah, with tashkeel marks
* aRABIC_TEXT_WITHOUT_TASHKEEL = the text of the aayah, without tashkeel marks
* tAFSEER_SADI_ARABIC = the tafseer of the aayah from as-sa'di
\end{comment}
\begin{comment}
The following tags are declared here:-
quran_com_link
quran_com_arabic_image
arabic_text_tashkeel
our_translation
variations_in_reading
translator_comments
aayah_tags
tafseer_sadi_arabic
tafseer_sadi_translation
license_attribution_aayah
\end{comment}
\begin{taggedblock}{quran_com_link}
\href{http://quran.com/49/9}{Link to 49:9 on quran.com}
\end{taggedblock}
\begin{taggedblock}{quran_com_arabic_image}
\includegraphics{49_9}
\end{taggedblock}
\begin{taggedblock}{arabic_text_tashkeel}
\begin{Arabic}
وَإِنْ طَائِفَتَانِ مِنَ الْمُؤْمِنِينَ اقْتَتَلُوا فَأَصْلِحُوا بَيْنَهُمَا فَإِنْ بَغَتْ إِحْدَاهُمَا عَلَى الْأُخْرَىٰ فَقَاتِلُوا الَّتِي تَبْغِي حَتَّىٰ تَفِيءَ إِلَىٰ أَمْرِ اللَّهِ فَإِنْ فَاءَتْ فَأَصْلِحُوا بَيْنَهُمَا بِالْعَدْلِ وَأَقْسِطُوا إِنَّ اللَّهَ يُحِبُّ الْمُقْسِطِينَ
\end{Arabic}
\end{taggedblock}
\begin{taggedblock}{our_translation}
وإن طائفتان من المؤمنين اقتتلوا فأصلحوا بينهما فإن بغت إحداهما على الأخرى فقاتلوا التي تبغي حتى تفيء إلى أمر الله فإن فاءت فأصلحوا بينهما بالعدل وأقسطوا إن الله يحب المقسطين
\end{taggedblock}
\begin{taggedblock}{variations_in_reading}
%This section is optional, for translating different wordings. For each different wording, paste the translation again, with the changes from Hafṣ highlighted in bold.
\end{taggedblock}
\begin{taggedblock}{translator_comments}
%Put any comments that you have as a translator, including issues of concern, or major decisions that you made when translating.
\end{taggedblock}
\begin{taggedblock}{aayah_tags}
%Put tags here separated by commas, e.g.: tawheed,prophets,yusuf,dua
\end{taggedblock}
\begin{taggedblock}{tafseer_sadi_arabic}
\begin{Arabic}
هذا متضمن لنهي المؤمنين،
[عن]
أن يبغي بعضهم على بعض، ويقاتل  بعضهم بعضًا، وأنه إذا اقتتلت طائفتان من المؤمنين، فإن على غيرهم من المؤمنين أن يتلافوا هذا الشر الكبير، بالإصلاح بينهم، والتوسط بذلك على أكمل وجه يقع به الصلح، ويسلكوا الطريق الموصلة إلى ذلك، فإن صلحتا، فبها ونعمت، وإن
{ بَغَتْ إِحْدَاهُمَا عَلَى الْأُخْرَى فَقَاتِلُوا الَّتِي تَبْغِي حَتَّى تَفِيءَ إِلَى أَمْرِ اللَّهِ }
أي: ترجع إلى ما حد الله ورسوله، من فعل الخير وترك الشر، الذي من أعظمه، الاقتتال،
[وقوله]

{ فَإِنْ فَاءَتْ فَأَصْلِحُوا بَيْنَهُمَا بِالْعَدْلِ }
هذا أمر بالصلح، وبالعدل في الصلح، فإن الصلح، قد يوجد، ولكن لا يكون بالعدل، بل بالظلم والحيف على أحد الخصمين، فهذا ليس هو الصلح المأمور به، فيجب أن لا يراعى أحدهما، لقرابة، أو وطن، أو غير ذلك من المقاصد والأغراض، التي توجب العدول عن العدل،
{ إِنَّ اللَّهَ يُحِبُّ الْمُقْسِطِينَ }
أي: العادلين في حكمهم بين الناس وفي جميع الولايات، التي تولوها، حتى إنه، قد يدخل في ذلك عدل الرجل في أهله، وعياله، في أدائه حقوقهم، وفي الحديث الصحيح:
"المقسطون عند الله، على منابر من نور الذين يعدلون في حكمهم وأهليهم، وما ولوا"
\end{Arabic}
\end{taggedblock}
\begin{taggedblock}{tafseer_sadi_translation}
هذا متضمن لنهي المؤمنين،
[عن]
أن يبغي بعضهم على بعض، ويقاتل  بعضهم بعضًا، وأنه إذا اقتتلت طائفتان من المؤمنين، فإن على غيرهم من المؤمنين أن يتلافوا هذا الشر الكبير، بالإصلاح بينهم، والتوسط بذلك على أكمل وجه يقع به الصلح، ويسلكوا الطريق الموصلة إلى ذلك، فإن صلحتا، فبها ونعمت، وإن
{ بَغَتْ إِحْدَاهُمَا عَلَى الْأُخْرَى فَقَاتِلُوا الَّتِي تَبْغِي حَتَّى تَفِيءَ إِلَى أَمْرِ اللَّهِ }
أي: ترجع إلى ما حد الله ورسوله، من فعل الخير وترك الشر، الذي من أعظمه، الاقتتال،
[وقوله]

{ فَإِنْ فَاءَتْ فَأَصْلِحُوا بَيْنَهُمَا بِالْعَدْلِ }
هذا أمر بالصلح، وبالعدل في الصلح، فإن الصلح، قد يوجد، ولكن لا يكون بالعدل، بل بالظلم والحيف على أحد الخصمين، فهذا ليس هو الصلح المأمور به، فيجب أن لا يراعى أحدهما، لقرابة، أو وطن، أو غير ذلك من المقاصد والأغراض، التي توجب العدول عن العدل،
{ إِنَّ اللَّهَ يُحِبُّ الْمُقْسِطِينَ }
أي: العادلين في حكمهم بين الناس وفي جميع الولايات، التي تولوها، حتى إنه، قد يدخل في ذلك عدل الرجل في أهله، وعياله، في أدائه حقوقهم، وفي الحديث الصحيح:
"المقسطون عند الله، على منابر من نور الذين يعدلون في حكمهم وأهليهم، وما ولوا"
\end{taggedblock}
\begin{taggedblock}{license_attribution_aayah}
\input{license-attribution}
\end{taggedblock}
\begin{comment}
Please use the following for footnotes:- Sample\footnoteQ{Text of Qur'an footnote goes here.}.
Sample\footnoteT{Text of Tafseer footnote goes here.}.
\end{comment}